%!TEX root = kdv.tex
%%%%%%%%%%%%%%%%%%%%%%%
%\section{Well-posedness of the problem}
%%%%%%%%%%%%%%%%%%%%%%%
\section{Well-posedness of the state equation}\label{secwellposedness}
In this section we discuss the time-global well-posedness of the state equation for irregular sources from $L^2(I,H^{-1}(\Omega))$ which includes $\M$ according to the last section. {\color{red} For the proof of the time-global well-posedness we need to distinguish the viscous case $\gamma>0$ and the non-viscous case $\gamma=0$. In the non-viscous case we are restricted to small data whereas in the viscous case no such constraints on the data are necessary.} First we present some necessary results concerning the linear \KdVB equation.
\subsection{Well-posedness of the linear \KdVB equation for {\color{red}$\gamma\geq 0$}}
First we consider
\begin{subequations}\label{kdvlinnonhom}
\begin{numcases}{}
\partial_t y +\partial_x y + \partial_{xxx} y - \gamma \partial_{xx} y =  f \mbox{ in } I\times\Omega,\label{kdvlinnonhom1}\\
y(\cdot,0) = y(\cdot,L) = \partial_x y (\cdot,L) = 0 \mbox{ in } I,\label{kdvlinnonhom2}\\
y(0,\cdot) = y_{0}(\cdot) \mbox{ in } \Omega,\label{kdvlinnonhom3}
\end{numcases}{}
\end{subequations}
and its dual counter part
\begin{subequations}\label{kdvlinnonhomdual}
\begin{numcases}{}
-\partial_t p -\partial_x p - \partial_{xxx} p - \gamma \partial_{xx} p =  \phi \mbox{ in } I\times\Omega,\label{kdvlinnonhomdual1}\\
p(\cdot,0) = p(\cdot,L) = \partial_x p (\cdot,0) = 0 \mbox{ in } I,\label{kdvlinnonhomdual2}\\
p(T,\cdot) = p_{T}(\cdot) \mbox{ in } \Omega,\label{kdvlinnonhomdual3}
\end{numcases}{}
\end{subequations}
where $\gamma\geq0$ is assumed. So we consider the the viscous and non-viscous case at the same time. We denote by $A\colon L^2(\Omega)\rightarrow L^2(\Omega)$ the linear differential operator
\[
Aw = -\partial_{xxx}w - \partial_{x}w + \gamma \partial_{xx}w
\]
with the dense domain $\mathcal{D}(A)\subset L^{2}(\Omega)$ defined by
\[
\mathcal{D}(A) = \left\{w\in H^{3}(\Omega) \mbox{ s.t. } w(0) = w(L) = \partial_xw(L) = 0\right\}.
\]
The adjoint operator $A^*\colon L^2(\Omega)\rightarrow L^2(\Omega)$ is given by
\[
A^*w = \partial_{xxx}w + \partial_{x}w + \gamma \partial_{xx}w
\]
with the domain
\[
\mathcal{D}(A^*) = \left\{w\in H^{3}(\Omega) \mbox{ s.t. } w(0) = w(L) = \partial_xw(0) = 0\right\}.
\]
The operators $A$ and $A^*$ generate strongly continuous semigroups of contractions on $L^{2}(\Omega)$ denoted by $W(\cdot)\colon L^2(\Omega)\rightarrow L^2(\Omega)$ and $W^*(\cdot)\colon L^2(\Omega)\rightarrow L^2(\Omega)$. The reader is referred to \cite{rosier1997exact} for a proof.  In the sequel, we will denote by $\mathcal{B}$ the Banach space
$C(I,L^2(\Omega))\cap L^2(I,H^1_0(\Omega))$ endowed with the norm
\[
\norm{y}_{\mathcal{B}} = \norm{y}_{\mathcal C(\bar I,L^2(\Omega))}+ \norm{y}_{L^2(I,H^1_0(\Omega))},\quad y\in \mathcal B
\]
where $\|\cdot\|_{H^1_0(\Omega)}=\|\partial_x \cdot\|_{L^2(\Omega)}$. {\color{blue} Furthermore we introduce the Hilbert space
\[
\mathcal V=H^2_0(\Omega)=\{v\in H^2(\Omega)\cap H^1_0(\Omega)\colon \partial_xv(0)=\partial_xv(L)=0\}
\]
endowed with the norm $\|\cdot\|_{\mathcal V}=\|\partial_{xx}\cdot\|_{L^2(\Omega)}$. Thus we have $\mathcal V^*=H^{-2}(\Omega)$.}
The following existence and uniqueness result can be found in \cite[Section 2]{BonaSunZhang03}.
\begin{proposition}\label{prop:ex smooth}
Let $(f,y_0)\in L^1(I,L^2(\Omega))\times L^2(\Omega)$ and $(\phi,p_T)\in L^1(I,L^2(\Omega))\times L^2(\Omega)$. Then equations \eqref{kdvlinnonhom1}-\eqref{kdvlinnonhom3} have a unique (mild) solution $y\in \mathcal B$ which is given by
\[
y(t)=W(t)y_0+\int_0^tW(t-s)f(s)~\mathrm ds\quad\forall t\in I
\]
and there exists a constant $c>0$ independent of $y_0$, $f$ and $y$ such that
\[
\|y\|_{\mathcal B}\leq c\,(\|f\|_{L^1(I,L^2(\Omega))}+\|y_0\|_{L^2(\Omega)})
\]
holds. Furthermore equations \eqref{kdvlinnonhomdual1}-\eqref{kdvlinnonhomdual3} have a unique solution $p\in \mathcal B$
given by
\be
p(t)=W^*(T-t)p_T+\int_t^TW^*(s-t)\phi(s)~\mathrm ds\quad\forall t\in I
\label{adjointmild}
\ee
and there exits a constant $c>0$ such that
\[
\|p\|_{\mathcal B}\leq c\,(\|\phi\|_{L^1(I,L^2(\Omega))}+\|p_T\|_{L^2(\Omega)})\]
holds.
\end{proposition}

Next we introduce a weak formulation of \eqref{kdvlinnonhom} for sources $f\in \Hm1$. %Since we are in a one-dimensional setting, this includes sources in $\M$.
\begin{definition}
For $(f,y_0)\in \Hm1\times L^2(\Omega)$ a function $y\in C(\bar I,L^2(\Omega))$ is called a weak solution of \eqref{kdvlinnonhom1}-\eqref{kdvlinnonhom3} if it satisfies the following equation
\begin{equation}\label{weakformlinearkdv}
\int_0^T(y,\phi)_{L^2(\Omega)}~\mathrm dt+(y(T),p_T)_{L^2(\Omega)}=\int_0^T\langle f,p\rangle_{H^{-1}(\Omega),H^1_0(\Omega)}~\mathrm dt+(y_0,p(0))_{L^2(\Omega)}
\end{equation}
for all $(\phi,p_T) \in L^1(I,L^2(\Omega))\times L^2(\Omega)$, where $p = p(\phi,p_T)\in \mathcal B$ is the mild solution of \eqref{kdvlinnonhomdual1}-\eqref{kdvlinnonhomdual3}.
\end{definition}

%The existence proof can be based on the transposition method using the linear operator of the dual equation defined in Proposition \ref{prop:ex smooth}, c.f. \cite[Part 2, section 2.2]{bensoussan07}. This would produce a unique solution $y\in L^{\infty}(I,L^2(\Omega))$ with $y(T)\in L^2(\Omega)$.
In order to show existence of a weak solution we utilize a strategy based on the approximation of the data.
\begin{proposition}\label{propnonhomo}
Let $(f,y_0)\in \Hm1\times L^2(\Omega)$. Then, there exists a unique weak
solution $y\in \mathcal B\cap H^1(I,\mathcal V^*)$  of \eqref{kdvlinnonhom1}-\eqref{kdvlinnonhom3}. Furthermore there exists a constant
$C(T,L) > 0$ such that the following estimate holds
\be\label{linestimate}
\norm{y}_{\mathcal B}+\|\partial_ty\|_{L^2(I,\mathcal V^*)}\leq C(T,L) \left(\norm{y_{0}}_{L^{2}(\Omega)} + \norm{f}_{\Hm1}\right).
\ee
\end{proposition}
\begin{proof}
We choose sequences
\begin{itemize}
  \item $\{f_n\}_{n\in\mathbb{N}}\subset\mathcal C^1(\bar I,L^2(\Omega))$ with $f_n\rightarrow f$ in $\Hm1$
  \item $\{y_{0,n}\}_{n\in\mathbb{N}}\subset\mathcal D(A)$ with $y_{0,n}\rightarrow y_0$ in $L^2(\Omega)$
\end{itemize}
which exist due to density. According to \cite[Part 2, Proposition 3.3]{bensoussan07} it exists a unique classical solution
\[y_n\in \mathcal C(\bar I,\mathcal D(A))\cap \mathcal C^1(\bar I,L^2(\Omega))\]
of \eqref{kdvlinnonhom} for data $f_n$ and $y_{0,n}$ which satisfies the weak form \eqref{weakformlinearkdv}. Furthermore it can be shown that $y_n$ satisfy the following estimate
\be
  \|y_n\|_{\mathcal B}\leq C(T,L) \left(\norm{y_{n,0}}_{L^{2}(\Omega)} + \norm{f_n}_{\Hm1}\right).
  \label{linestimate_regular}
\ee
For a proof see \cref{sec:linear-estimates}.  This estimate implies that $\{y_n\}_{n\in \mathbb{N}}$ is a Cauchy sequence in $\mathcal B$ and therefore there exists a $y\in \mathcal B$ which satisfies \eqref{weakformlinearkdv} with the data $(f,y_0)$. This means that $y$ is a weak solution of \eqref{kdvlinnonhom}. Its uniqueness can be shown using standard arguments. Furthermore we can  pass to the limit in \eqref{linestimate_regular} and get the first part of \eqref{linestimate}. Next we choose any {\color{blue} $\psi\in \mathcal C_c^{\infty}(I\times \Omega)$} and set $\phi=\partial_t\psi-A^*\psi$ in \eqref{kdvlinnonhomdual}. Therefore $\psi$ is the solution of \eqref{kdvlinnonhomdual1}-\eqref{kdvlinnonhomdual3} and it holds
\begin{multline*}
\int_0^T(y,\partial_t\psi)_{L^2(\Omega)}~\mathrm dt=\int_0^T(y,A^*\psi)_{L^2(\Omega)}+\langle f,\psi\rangle_{H^{-1}(\Omega),H^1_0(\Omega)}\mathrm dt\\
\leq C(T,L)\left(\|y\|_{L^2(I,H^1_0(\Omega))}+\|f\|_{L^2(I,H^{-1}(\Omega))}\right)\|\psi\|_{L^2(I,\mathcal V)}.
\end{multline*}
{\color{blue} Due to the density of $\mathcal C_c^\infty(I\times \Omega)$ in $L^2(I,\mathcal V)$, there holds $y\in H^1(I,\mathcal V^*)$ and
\[\|\partial_t y\|_{L^2(I,\mathcal V^*)}\leq C(T,L)\left(\|f\|_{L^2(I,H^{-1}(\Omega))}+\|y_0\|_{L^2(\Omega)}\right).\]}
\qquad\end{proof}

\begin{definition}\label{rmklinearoperator}
{\color{red}
The operator
\[
\mathcal{L}:\Hm1\times L^2(\Omega)\rightarrow \mathcal{B},(f,y_0) \mapsto y,
\]
where $y$ is the weak solution of \eqref{kdvlinnonhom1} - \eqref{kdvlinnonhom3}, is called solution operator of \eqref{kdvlinnonhom1} - \eqref{kdvlinnonhom3}. It is linear and bounded.}
\end{definition}
\subsection{Time-local well-posedness of the \KdVB equation with $\gamma\geq0$}
{\color{red} We consider in this section the nonlinear \KdVB equation \eqref{kdvcontrol1} - \eqref{kdvcontrol3} with sources from $f\in \Hm1$ and show its time-local wellposedness in the viscous and non-viscous case, i.e. $\gamma\geq0$.} First of all we introduce a suitable solution concept for the \KdVB equation
\begin{subequations}
\begin{numcases}{}
\partial_t y +\partial_x y + \partial_{xxx} y + y\partial_x y -\gamma \partial_{xx} y=  f \mbox{ in } I\times\Omega,\label{kdv1}\\
y(.,0) = y(.,L) = \partial_x y (.,L) = 0,\label{kdv2}\mbox{ in } I,\\
y(0,.) = 0 \mbox{ in } \Omega\label{kdv3},
\end{numcases}
\end{subequations}
for sources from $\Hm1$.
\begin{definition}\label{defnlkdv}
For $(y_0,f)\in L^2(\Omega)\times \Hm1$ a function $y\in \mathcal B$ is called a weak solution of \eqref{kdv1} - \eqref{kdv3} if it satisfies the following fixed point equation
\[y=\mathcal L(f-y\partial_x y ,y_0),\]
or in other words
\begin{equation}\label{weakformkdv}
\int_0^T(y,\phi)_{L^2(\Omega)}~\mathrm dt+(y(T),p_T)_{L^2(\Omega)}=\int_0^T\langle f-y\partial_x y ,p\rangle_{H^{-1}(\Omega),H^1_0(\Omega)}~\mathrm dt+(y_0,p(0))_{L^2(\Omega)}
\end{equation}
for all $(\phi,p_T) \in L^1(I,L^2(\Omega))\times L^2(\Omega)$, where $p(\phi,p_T)\in \mathcal B$ is the solution of \eqref{kdvlinnonhomdual1}-\eqref{kdvlinnonhomdual3}.
\end{definition}

The last definition makes sense considering the next Lemma which is also needed for the proof of existence of a solution to \eqref{kdv1} - \eqref{kdv3}.
\begin{lemma}\label{lemyyx2}
 Let $T > 0$, $y \in \B$ and $z \in \B$, then there exists a $c>0$ independent of $T$ such that
 \[
 \norm{y \partial_x y - z \partial_x z}_{\Hm1} \leq c\, T^{1/4} \norm{y+z}_{\B} \norm{y - z}_{\B}.
 \]
\end{lemma}
%[Proof of Lemma~\ref{lemyyx2}]
\begin{proof} The proof is provided in \cref{sec:nonl-state-equat} and is largely inspired from \cite{faminskii2010initial}.
\qquad\end{proof}

Let us define for an arbitrary $\theta \leq T$ the space
\be
\mathcal{B}_{\theta} =  C([0,\theta],L^2(\Omega))\cap L^2((0,\theta), H^1_0(\Omega)),
\label{btheta}
\ee
endowed with the norm
\be
\norm{y}_{\mathcal{B}_{\theta}} = \norm{y}_{C([0,\theta], L^2(\Omega))} + \norm{y}_{L^2([0,\theta], H^1_0(\Omega))}.
\label{normbtheta}
\ee
\begin{proposition}\label{localposedness}
For any $f \in \Hm1$ and $y_0\in L^2(\Omega)$, there exists a $T^{\ast} \in [0,T]$ depending on $\norm{f}_{L^{2}(I, H^{-1}(\Omega))}$ and $\|y_0\|_{L^2(\Omega)}$ such that the system \eqref{kdv1} - \eqref{kdv3} admits a unique weak solution $y\in \mathcal B_{T^*}$ which satisfies \eqref{weakformkdv} with $T=T^\ast$.
\end{proposition}
%[Proof of Proposition~\ref{localposedness}]
\begin{proof}
For $\theta \in (0,T]$, we define the operator $\Psi_{f,y_0}^\theta : \mathcal{B}_{\theta} \mapsto \mathcal{B}_{\theta}$ as
\be
%\Psi_{q,y_0}(z) = W_0(t)y_0-\int_0^t{W_0(t-s)(z\partial_x z)(s,.)ds} + \int_0^t{W_0(t-s)q(s,.)ds}
\Psi_{f,y_0}(z) = \mathcal{L}(f-z\partial_x z,y_0).
\label{operatorBanach}
\ee
which is the weak solution of \eqref{kdvlinnonhom1}-\eqref{kdvlinnonhom3} with $T=\theta$ for the data $(f-z\partial_x z,y_0)$.
Estimate \eqref{linestimate} and \cref{lemyyx2} imply
\begin{multline}
\norm{\Psi_{f,y_0}(y)}_{\mathcal{B}_{\theta}} \leq C_1 \left(\norm{y_0}_{L^2(\Omega)} + \norm{f}_{L^2(I,H^{-1}(\Omega))}+\norm{y\partial_x y}_{L^2(I,H^{-1}(\Omega))}\right)\\
\leq C_1 \left(\norm{y_0}_{L^2(\Omega)} + \norm{f}_{L^2(I,H^{-1}(\Omega))}\right) + C_2\theta^{1/4}\norm{y}_{\mathcal{B}_{\theta}}^2
\label{normpsi2}
\end{multline}
and
\[
\norm{\Psi_{f,y_0}(y) - \Psi_{f,y_0}(z)}_{\mathcal{B}_{\theta}} \leq C_2 \theta^{1/4} \norm{y+z}_{\mathcal{B}_{\theta}}\norm{y - z}_{\mathcal{B}_{\theta}}.
\label{diffpsi2}
\]
We choose $\theta > 0$ such that
\begin{subequations}
 \begin{numcases}{}
  r = 3 C_1 \left(\norm{y_0}_{L^2(\Omega)} + \norm{f}_{L^{2}(0,T, H^{-1}(\Omega))}\right)\label{constraintstheta1}\\
  C_2 \theta^{1/4} r \leq \frac{1}{3} \label{constraintstheta2}
 \end{numcases}
\end{subequations}
holds. Therefore, by considering the ball $B = \{ y \in \mathcal{B}_{\theta}; \norm{y}_{\mathcal{B}_{\theta}} \leq r\}$ we have
\[
\Psi_{f,y_0}(B) \subset B
\]
and for all $(y,z) \in B$
\[
\|\Psi_{f,y_0}(y) - \Psi_{f,y_0}(z)\|_{\mathcal{B}_{\theta}}\leq \frac{2}{3}\norm{y - z}_{\mathcal{B}_{\theta}}.
\]
As a consequence, we can apply the Banach fixed point theorem which implies the existence of a unique fix point $y$ of $\Psi_{f,y_0}(\cdot)$.
\qquad\end{proof}
According to the proof of \cref{localposedness}, an upper bound for $T^{\ast}$ is given by
\[
T^{\ast}\leq \frac{C(T,L)}{\left( \norm{y_0}_{L^2(\Omega)} + \norm{f}_{L^2(I,H^{-1}(\Omega))}\right)^{4}}.
\]
The bigger $\|f\|_{\Hm1}$, the shorter we can ensure the existence of the solution.
\subsection{Time-global well-posedness of the \KdVB equation {\color{red}with $\gamma=0$ and small data}}\label{rmkUad}
{\color{blue} In this section we establish the time-global wellposedness of the \KdVB equation in the non-viscous case $\gamma =0$ and with small data. The inequality \eqref{constraintstheta2} can be also satisfied for a fixed $T$ and small data, i.e. if
\be
\norm{f}_{L^2(I,H^{-1}(\Omega))}+\|y_0\|_{L^2(\Omega)} \leq \frac{C(T,L)}{T^{1/4}}.
\ee
is satisfied.
%Since we will only be able to prove global well-posedness in the case $\gamma > 0$, we will rather adopt this point of view as far as optimization is concerned for the pure \KdV equation.
In particular, we have the following result:
\begin{corollary}\label{existlocal}
Let $\gamma=0$. Then there exists a constant $C_{\max}(T,L)>0$ such that for any  $(f,y_0) \in \Hm1\times L^2(\Omega)$ with
\be\label{ineqqnorm}
\norm{f}_{L^2(I,H^{-1}(\Omega))}+\|y_0\|_{L^2(\Omega)} \leq C_{\max}
\ee
there exists a unique solution $y\in \mathcal B$ of \eqref{kdv1} - \eqref{kdv3} which satisfies
\be\label{localestimate}
\norm{y}_{\mathcal B}\leq c\left( \norm{y_0}_{L^2(\Omega)} + \norm{f}_{L^2(I,H^{-1}(\Omega))}\right)
\ee
for some $c(T,L)>0$ independent of $y$ and the data. Moreover, there holds
\begin{equation}\label{estyt}
\|\partial_t y\|_{L^2(I,\mathcal V^\ast)}\leq \tilde c(L,T,y_0,f)
\end{equation}
for some constant $\tilde c>0$ only depending on $L,T,y_0,f$.
\end{corollary}
\begin{proof}
Existence, uniqueness and \eqref{localestimate} are proven using similar steps as in the proof of \cref{localposedness}. In particular, the inequality \eqref{constraintstheta2} is satisfied for our choice of data $(u,y_0)$.  Now we proof the estimate \eqref{estyt} using \eqref{linestimate}, \eqref{localestimate} and \cref{lemyyx2}. In particular, we have:
\begin{align*}
\|\partial_ty\|_{L^2(I,\mathcal V^\ast)}&\leq c\left(\|y_0\|_{L^2(\Omega)}+\|y\partial_xy\|_{\Hm1}+\|f\|_{\Hm1}\right)\\
&\leq c\left(\|y_0\|_{L^2(\Omega)}+\|y\|_{\mathcal B}^2+\|f\|_{\Hm1}\right)\\
&\leq \tilde c(T,L,y_0,f)
\end{align*}
\qquad\end{proof}}

{\color{red}\subsubsection{Time-global well-posedness of the \KdVB equation with $\gamma>0$ and general data}
In this section we prove time-global wellposedness of the \KdVB equation in the viscous case and with general data. \Cref{localposedness} guarantees local well-posedness in time. Therefore, global well-posedness will follow from a priori estimates for the nonlinear problem on $[0,T]$. We will show that such estimates exist in the viscous case $\gamma > 0$.
\begin{theorem}
Let $\gamma>0$. For any  $(f,y_0) \in \Hm1\times L^2(\Omega)$ let $y\in\mathcal B_\theta$, $\theta\in (0,T]$, be a time-local solution of \eqref{kdv1}-\eqref{kdv3}, then $y$ satisfies
 \be
 \norm{y}_{\mathcal B_\theta}\leq c\,\left( \frac{\sqrt{\gamma}+1}{\sqrt{\gamma}}\right) \left(\norm{y_{0}}_{L^{2}(\Omega)} + \frac{1}{\sqrt{\gamma}}\norm{f}_{L^2(I,H^{-1}(\Omega))}\right)
 \label{globalestimate}
 \ee
 for some $c>0$ independent of $y$ and the data.
\end{theorem}
\begin{proof}
We consider first that $y$ is a classical solution $y\in \mathcal C([0,\theta],\mathcal D(A))\cap \mathcal\, C^1([0,\theta],L^2(\Omega))$ for smooth data which exists according to \cite{faminskii2010initial}. Then equation \eqref{kdv1} holds in $L^2(\Omega)$ and we can multiply it with $y$ which yields
\[
\frac{1}{2}\frac{d}{dt}\|y(t)\|_{L^2(\Omega)}^2 + \frac 1 2\abs{\partial_{x} y(t,0)}^{2} + \gamma \|y(t)\|_{H^1_0(\Omega)}^2= \langle f(t),y(t)\rangle_{H^{-1}(\Omega),H^{1}_{0}(\Omega)}
\]
since
\[
\int_\Omega y\partial_x y~\mathrm dx=\int_\Omega y^2\partial_x y~\mathrm dx=0.
\]
Using the continuity of the duality pairing on the righthand side and Young's inequality, we get
\[
\frac{1}{2}\frac{d}{dt}\|y\|_{L^2(\Omega)}^2 + \frac 1 2\abs{\partial_{x} y(t,0)}^{2} +  \gamma \|y\|_{H^1_0(\Omega)}^2\leq \frac{1}{2\gamma}\norm{f}_{H^{-1}(\Omega)}^{2} + \frac{\gamma}{2}\norm{y}_{H^{1}_{0}(\Omega)}^{2}
\label{linnhupperbound1}.
\]
Then we get by omitting the positive term $\abs{\partial_{x} y(t,0)}^{2}$ and by subtraction of $\gamma/2\norm{y}_{H^{1}_{0}(\Omega)}^{2}$ the following inequality
\[
\frac{1}{2}\frac{d}{dt}\|y\|_{L^2(\Omega)}^2 + \frac{\gamma}{2} \|y\|_{H^1_0(\Omega)}^2\leq \frac{1}{2\gamma}\norm{f}_{H^{-1}(\Omega)}^{2}.
\label{estimatenonlin}
\]
Omitting the term $\frac{\gamma}{2} \|y\|_{H^1_0(\Omega)}^2$ and integration between $0$ and $t\leq \theta$ yields
\be
\norm{y(t,\cdot)}_{L^2(\Omega)}^2 \leq \norm{y_0}_{L^2(\Omega)}^2+\frac 1\gamma \int_0^t\norm{f(s)}_{H^{-1}(\Omega)}^{2}~\mathrm ds\leq \norm{y_0}_{L^2(\Omega)}^2 + \frac{1}{\gamma}\norm{f}_{L^2(I,H^{-1}(\Omega))}^2.
\label{C0nonlin}
\ee
Similarity integration between $0$ and $\theta$ gives
\[
\norm{y(\theta,.)}_{L^2(\Omega)}^2 +  \gamma \norm{y}_{L^2((0,\theta),H^1_0(\Omega))}^{2} \leq \norm{y_0}_{L^2(\Omega)}^{2} + \frac{1}{\gamma}\norm{f}_{L^2(I,H^{-1}(\Omega))}^{2},
\label{H10nonlin}
\]
Omitting the term $\norm{y(\theta,.)}_{L^2(\Omega)}^2$ results in
\be
\norm{y}_{L^2((0,\theta),H^1_0(\Omega))}^{2} \leq \frac{1}{\gamma}\norm{y_0}_{L^2(\Omega)}^{2} + \frac{1}{\gamma^2}\norm{f}_{L^2(I,H^{-1}(\Omega))}^{2}.
\label{H10nonlin2}
\ee
Adding \eqref{C0nonlin} and \eqref{H10nonlin2} gives the global estimate \eqref{globalestimate} for smooth data which can be extended by density to $\Hm1\times L^2(\Omega)$ using \cref{prop:tangent}.
\qquad\end{proof}}

\begin{remark}
The constant in the estimate \eqref{globalestimate} explodes for $\gamma \rightarrow 0$.
\end{remark}
{\color{blue}
\begin{corollary}
Let $\gamma >0$ and $(f,y_0) \in \Hm1\times L^2(\Omega)$. Then there exists a unique solution $y\in \mathcal B\cap L^2(I,\mathcal V^\ast)$ of \eqref{kdv1}-\eqref{kdv3} which satisfies
\be
 \norm{y}_{\mathcal B}\leq c(\gamma) \left(\norm{y_{0}}_{L^{2}(\Omega)} + \norm{f}_{L^2(I,H^{-1}(\Omega))}\right)
\ee
for some $c>0$ independent of $y$ and the data. Moreover there holds
\begin{equation}\label{estytglobal}
\|\partial_t y\|_{L^2(I,\mathcal V^*)}\leq C(T,L,y_0,f,\gamma).
\end{equation}
for some $C>0$ only depending on $L,T,y_0,f$.
\end{corollary}
\begin{proof}
First, we discuss uniqueness. Let $y_1$ and $y_2$ two solutions of \eqref{kdv1}-\eqref{kdv3}. We define $\delta y=y_1-y_2$. Then $\delta y$ solves equation \eqref{linkdv1}--\eqref{linkdv3} with $y=y_1+y_2$. Then \cref{prop:tangent} implies $y_1=y_2$. According to \cref{localposedness} equation \eqref{linkdv1}--\eqref{linkdv3} has a unique time-local solution $y\in \mathcal B_{T^\ast}$. Thus we can repeat the procedure of \cref{localposedness} starting from $y(T^\ast)$ until a maximal time of existence $0<t_{\max}\leq T$ is reached. If $t_{\max}<T$ it follows from \cite[Chapter 6, Theorem 2.2]{pazy1983semigroups} that $\lim_{t\uparrow t_{\max}}\|y(t)\|_{L^2(\Omega)}=\infty$ which contradicts \eqref{globalestimate}. The estimate \eqref{estytglobal} is proven using \eqref{globalestimate} and similar arguments as in the proof of \cref{existlocal}.
\qquad\end{proof}}

\section{Well-posedness of the optimization problem. Existence of an optimum}\label{sec:ex opt}
We start by defining the admissible set of controls
$$U_{ad} = \begin{cases} \left\{u \in \M\colon \norm{u}_{\M}\leq c_{ad} \right\} &\quad\mbox{if } \gamma = 0 \\
\M & \quad \mbox{otherwise}
 \end{cases}$$
where $c_{ad}< C_{\max}-\norm{y_0}_{L^2(\Omega)}$. The constant $C_{\max}$ comes from \eqref{ineqqnorm} and we assume that
\begin{equation}\label{constr y0}
\norm{y_0}_{L^2(\Omega)}< C_{\max}.
\end{equation}
Moreover we introduce the non-linear control-to-state operator
\begin{equation}
 S\colon U_{ad}\rightarrow \mathcal B\cap H^1(I,\mathcal V^\ast),\quad u\mapsto y
 \label{controltostate}
\end{equation}
where $y$ is a weak solution of \eqref{kdvcontrol1}-\eqref{kdvcontrol3} for a given $y_0\in L^2(\Omega)$ which satisfies
\eqref{constr y0} in the case $\gamma =0$. According to \cref{rmkUad} the mapping $S$ is well-defined. In \cite{ClasonKaltenbacher13} the authors used a similar strategy for the definition of the control-to-state mapping. The control-to-observation operator is denoted by
\[
S_{obs}\colon U_{ad}\rightarrow L^2(I\times \Omega_{o})\times L^2(\Omega_{o}),\quad u\mapsto(\chi_{\Omega_{o}}y,\chi_{\Omega_{o}}y(T)).
\]
Thus we can rewrite problem \eqref{cost} in its reduced form given by
\be
\min_{u \in U_{ad}}\frac{1}{2}\norm{S_{obs}(u) - z}_{L^2(I\times \Omega_{o})\times L^2(\Omega_{o})}^2 + \alpha \norm{u}_{\mathcal{M}(\Omega_{c}, L^{2}(I))}.
\label{red cost}
\ee
with $z\in L^2(I\times \Omega_{o})\times L^2(\Omega_{o})$.
\begin{proposition}
There exists a solution $(\bar y,\bar u) \in \mathcal B\times U_{ad}$ to the optimal control problem \eqref{cost}.
\end{proposition}
\begin{proof}
\underline{Case $\gamma > 0$}. The cost function $J$ is a positive function. Thus its infimum $\bar J$ exists and there exists a minimizing sequence $(u_n,y_n) \in U_{ad} \times \mathcal B$ such that $J(u_n, y_n) \rightarrow \bar J$ as $n \rightarrow \infty$. Furthermore there exists an $\varepsilon>0$ such that
\be
\varepsilon+J(0,0)\geq \alpha \norm{u_n}_{\M}
\ee
for $n$ large enough. Therefore $u_n$ is bounded, which implies the existence of an element $\bar u\in\M$ and a subsequence $u_{n_k}$ converging in the weak-$*$ topology of $\M$ towards $\bar u$, cf., \cite[Corollary 3.30]{Brezis11}. For each $u_{n_k}$, we define $y_{n_k}=Su_{n_k}\in \mathcal B\cap H^1(I,\mathcal V^*)$. Thanks to the global estimate \eqref{globalestimate}, there exists a exists a $\bar y\in L^2(I,H^1_0(\Omega))\cap\,L^\infty(I,L^2(\Omega))\cap\,H^1(I,\mathcal V^\ast)$ such that
\[y_{n_k}\rightharpoonup^\ast\bar y\quad\text{in}\quad L^2(I,H^1_0(\Omega))\cap L^{\infty}(I,L^2(\Omega))\cap H^1(I,\mathcal V^\ast)\]
and a $\hat y\in L^2(\Omega)$ such that $y_{n_k}(T)\rightharpoonup \hat y$ in $L^2(\Omega)$ up to a subsequence. Moreover the Aubin-Lions-Lemma, cf. \cite[Chapter 3, Proposition 1.3]{showalter97}, implies
\[y_{n_k}\rightarrow\bar y\quad\text{in}\quad L^2(I,\mathcal C_0(\Omega)).\]
It remains to show that the limit $\bar y$ is indeed a weak solution of \eqref{kdvcontrol1} - \eqref{kdvcontrol3} for the control $\bar u$. Due to the weak-to-weak continuity of time-point evaluation operator $y\mapsto y(T)$ from $H^1(I,\mathcal V^\ast)$ to $\mathcal V^\ast$ there holds $\hat y=y(T)$.  The convergence of the linear terms in \eqref{weakformkdv} is obvious. The nonlinear term converges due to the strong convergence of $y_{n_k}$, since
\begin{multline*}
\int_0^T\langle y_{n_k}\partial_xy_{n_k}-\bar y\partial_x\bar y,p\rangle_{H^{-1}(\Omega),H^1_0(\Omega)}~\mathrm dt=-\int_0^T(y_{n_k}^2-\bar y^2,\partial_x p)_{L^2(\Omega)}~\mathrm dt\\
\leq\|y_{n_k}-\bar y\|_{L^2(I,\mathcal C_0(\Omega))}\|y_{n_k}+\bar y\|_{\mathcal C(\bar I,L^2(\Omega))}\|\partial_x p\|_{L^2(I\times \Omega)}.
\end{multline*}
Therefore $\bar y$ is a weak solution of \eqref{kdvcontrol1}-\eqref{kdvcontrol3} for the control $\bar u$ and it holds $\bar y\in \mathcal B\cap H^1(I,\mathcal V^*)$. Moreover there holds
\[S_{obs}(u_{n_k})\rightharpoonup S_{obs}(\bar u)\quad\text{in}~ L^2(I\times \Omega_{o})\times L^2(\Omega_{o}).\] The tracking functional is weak lower semi-continuous in $ L^2(I\times \Omega_{o})\times L^2(\Omega_{o})$ and the control cost term is weak-$*$ lower semi continuous in $\M$ therefore $(\bar y,\bar u)$ is a solution of \eqref{cost}.\\
\underline{Case $\gamma = 0$}. In this case we rely on the estimate \eqref{localestimate} and use that the set $U_{ad}$ is weak-$*$ closed.
\qquad\end{proof}

\begin{remark}\label{alphaconstraint}
 We know for a minimizing sequence $(u_n, y_n)\in \M\times \mathcal B$ that there exists an $\varepsilon > 0$ and an $N \in \N $ such that for any $n > N$
 \[
\norm{u_n}_{L^2(I,H^{-1}(\Omega))} \leq c\norm{u_n}_{\M} \leq \frac{c}{\alpha}\left(J(0,0) + \varepsilon \right).
 \]
 Therefore, we can always find a regularization parameter $\alpha>0$ such that the condition \eqref{ineqqnorm} is satisfied.
\end{remark}

%%% Local Variables:
%%% mode: latex
%%% TeX-master: "kdv.tex"
%%% End: 