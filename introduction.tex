%!TEX root = kdv.tex

\section{Introduction}

The Korteweg-de Vries equation first appeared in 1895 in the context of water waves \cite{korteweg1895xli}. It was designed to model the evolution of long water waves in a channel of rectangular cross section when the effects of nonlinearity and dispersion balance. This phenomenon gives rise to the so-called soliton, a wave traveling at constant speed without losing its shape. This equation has been theoretically widely studied: much work has been devoted to the derivation of the equation from Euler equations \cite{shen1992forced,constantin2008,su2003korteweg}, but also to the proof of their well-posedness in various contexts \cite{miura1976korteweg,kenig1993,bourgain1997periodic} - periodic domain, on the real line, bounded domain -, or to their controllability \cite{rosier1997exact,glass2008some,coron2003exact,chapouly2009global}.
%Because many applications were investigated, and in particular, numerous works in environmental sciences but also life sciences - see \cite{dauxois2006physics,whitham2011linear,Crepeau2007594,yomosa1987} and the references therein, lots of numerical methods have been developped to solve it efficiently \cite{trefethen2000spectral,shen2003new,ma2000legendre}.
One application of the \KdV equation that is of particular interest for us is the modeling of a flow in a narrow channel over an obstacle \cite{milewski2004forced,shen1992forced,shen1996accuracy}. % be it more particularly a water wave over rocks or an atmospheric flow over a topographic obstacle \cite{baines1997topographic}.
In that case, a source term is added on the right-hand side of the \KdV equation, that represents the derivative of the topography under the flow, and the resulting equation is called the forced Korteweg-de Vries equation. More generally we consider the forced \KdVB-equation which describes the viscous flow over a topography. This is done on the one hand to include viscosity and on the other hand for mathematical reasons.      %Considering the various rescalings and asymptotics involved in the derivation of the \KdV equation, it is a reasonable assumption to model its effect by a Dirac delta function in space \cite{shen1996accuracy, shen2000bumpdirac}.
The idea of this paper is to provide a framework to tackle two kinds of problems regarding the Korteweg-de Vries equation: an inverse problem - are we able to reconstruct a time varying topography, e.g., the locations of jumps with varying height at the bottom of the channel, from the noisy observations of wave patterns ? -  and a control problem - is it possible to create a certain desired wave while acting on the topography ? As pointed out earlier, the considered topography is assumed to consist of jumps with time varying heights. Thus its derivative is a linear combination of Dirac measures with time-independent positions $x_i$ and time-dependent magnitudes $u_i$, i.e.,
\begin{equation}\label{lin comb dirac}
\sum_{i=1}^{N}{u_{i}(t)\delta_{x_{i}}(x)}.
\end{equation}
Then the control problem consists of finding the optimal positions and optimal time-depending heights of the jumps given either noisy measurements of the wave patterns or a desired wave. As pointed out in \cite{pieper2014} and \cite{KunischTrautmannVexler14} the space of vector measures $\M$ with values in $L^2(I)$ contains sources of the form \eqref{lin comb dirac} and the use of its norm $\|\cdot\|_{\M}$ as control cost functional enhances optimal controls of the form \eqref{lin comb dirac}. In this perspective, we follow the path introduced in \cite{pieper2014,KunischTrautmannVexler14} and focus on the optimal control problem
\begin{multline}
\min_{u \in \M, y\in Y}J(y,u)=\frac{1}{2}\left(\norm{\chi_{\Omega_{o}}y - z_1}_{L^2(I\times \Omega_{o})}^2+\|\chi_{\Omega_{o}}y(T)-z_2\|_{L^2(\Omega_{o})}^2\right)\\
+ \alpha \norm{u}_{\mathcal{M}(\Omega_c, L^{2}(I))}
\label{cost}
\end{multline}
with $(z_1,z_2)\in L^2(I\times \Omega_o)\times L^2(\Omega_o) $ where $y\in Y$ is the solution of the Korteweg-de Vries-Burgers equation with Dirichlet and Neumann boundary conditions on $\Omega = (0,L)$ (the space $Y$ will be defined later on)
\begin{subequations}
\begin{numcases}{}
\partial_t y +\partial_x y + \partial_{xxx} y + y\partial_x y -\gamma \partial_{xx} y=  u \mbox{ in } I\times\Omega,\label{kdvcontrol1}\\
y(\cdot,0) = y(\cdot,L) = \partial_x y (\cdot,L) = 0\mbox{ in } I,\label{kdvcontrol2}\\
y(0,\cdot) = y_0(\cdot) \mbox{ in } \Omega\label{kdvcontrol3}.
\end{numcases}
\end{subequations}
The control acts on the control domain $\Omega_c\subseteq \Omega$. The state variable $y$ is tracked on the observation domain $\Omega_{o}\subseteq\Omega$. The parameter $\alpha > 0$ is the control cost parameter while  $\gamma \geq 0$ is the viscosity parameter. %Denoting by $I=[0,T]$ the time horizon considered, the control variable $u$ lies in the space of Borel measures with values in $L^2(I)$ that we will denote $\M$.
%A crucial feature of our mathematical analysis is based on the fact that this space can be identified with the topological dual of $C_{0}(\Omega,L^2(I))$ \cite{clason2011duality,casas2012approximation}, i.e. the space of continuous functions with compact support in $\Omega$ and values in $L^2(I)$.
%To give an insight, $\M$ contains functions of the type $u(t,x) = \sum_{i=1}^{N}{u_{i}(t)\delta_{x_{i}}(x)}$, with $u_i \in L^2(I)$ and $\delta_{x_i}$ ard Dirac delta functions located at the fixed points $x_i$. But we want to stress that it does not include moving Dirac delta functions. Those functions would rather be elements of $L^2(I,\mathcal{M}(\Omega))$, with $\M \subset L^2(I,\mathcal{M}(\Omega))$.
Similar types of measure-valued control problems have already been studied in the case of linear elliptic and parabolic equations \cite{pieper2013priori,clason2011duality,casas2012approximation,ClasonKunisch:2011b} and \cite{casas2013parabolic,CasasZuazua13,CasasVexlerZuazua13,CasasKunisch15}.
Our approach is connected to \cite{herzog2012directional} where the authors use the control cost functional
\[
u\mapsto\alpha \|u\|_{L^1(\Omega_c,L^2(I))}+\frac \varepsilon 2 \|u\|_{L^2(I\times \Omega_c)}^2
\]
in connection with parabolic optimal control problems. This control cost functional promotes sparsity patterns of the optimal control which are constant in time (directional sparsity, joint sparsity). Our problem setting is equivalent to theirs for $\varepsilon=0$. Moreover we solve \eqref{cost} using a continuation/homotopie strategy involving the regularization term $(\varepsilon/2)\|\cdot\|_{L^2(I\times \Omega_c)}^2$ for $\varepsilon\rightarrow 0$.
%To the authors knowledge, optimal control of the Korteweg-de Vries-Burgers equation is still an open problem, especially in a sparsity promoting framework.
The mathematical challenges of this work are twofold: on the one hand we shall prove well-posedness of the state equation in the presence of an irregular source term while on the other hand sparse optimal control of a nonlinear dispersive partial differential equation is also a novel question. %We point out that the quantities $y$, $x$, and $t$ can be rescaled to produce any desired coefficients for the terms of \eqref{kdvcontrol1} - \eqref{kdvcontrol3}. The choice we make here is convenient for the mathematical analysis of the problem.

%%% OUTLINE
This paper is organized as follows. In Section \ref{control space} we discuss properties of the control space $\M$. Section~\ref{secwellposedness}, Section \ref{sec:ex opt} and Section~\ref{wp2} deal with the well-posedness of the optimal control problem with measure-valued controls. This includes a study of the forward PDE problem with irregular data. Section~\ref{secoptconditions} displays the optimality conditions of \eqref{cost}. Afterwards, we expose in Section~\ref{secnum} the various numerical strategies we adopt for the solution of the state equation and  of the optimization problem. We conclude with some numerical examples on the Korteweg-de Vries equation, which include: an inverse problem and a control problem.

%%% Local Variables:
%%% mode: latex
%%% TeX-master: "kdv.tex"
%%% End:
