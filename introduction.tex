%!TEX root = kdv.tex

\section{Introduction}

The Korteweg-de Vries equation first appeared in 1895 in the context of water waves \cite{korteweg1895xli}. It was designed to model the evolution of long water waves in a channel of rectangular cross section when the effects of nonlinearity and dispersion balance. This phenomenon gives rise to the so-called solition, a solitary wave traveling at constant speed without losing its shape. This equation has been theoretically widely studied: much work has been devoted to the derivation of both equations from Euler equations \cite{shen1992forced,constantin2008,su2003korteweg}, but also to the proof of their well-posedness in various contexts \cite{miura1976korteweg,kenig1993,bourgain1997periodic} - periodic domain, on the real line, bounded domain -, to their controllability \cite{rosier1997exact,glass2008some,coron2003exact,chapouly2009global}.
Because many applications were investigated, and in particular, numerous works in environmental sciences but also life sciences - see \cite{dauxois2006physics,whitham2011linear,Crepeau2007594,yomosa1987} and the references therein, lots of numerical methods have been developped to solve it efficiently \cite{trefethen2000spectral,shen2003new,ma2000legendre}. One application that is of particular interest for us is the modeling of a flow over an obstacle \cite{milewski2004forced,shen1992forced,shen1996accuracy}, be it more particularly a water wave over rocks or an atmospheric flow over a topographic obstacle \cite{baines1997topographic}. In that case, a source term is added on the right-hand side of the state equation, that represents the derivative of the topography under the flow, and the resulting equation is called the forced Korteweg-de Vries equation. Considering the various rescalings and asymptotics involved in the derivation of the \KdV equation, it is a reasonable assumption to model its effect by a Dirac delta function in space \cite{shen1996accuracy, shen2000bumpdirac}. The idea of this paper is to provide a framework to tackle two kinds of problems regarding the Korteweg-de Vries equation: an inverse problem - are we able to find the location and amplitude of a topographic bump from noisy observations ? -  and a control problem - is it possible to create a wave in finite time while acting on the topography ? As pointed out earlier, the topography should be mostly composed of jumps such that the derivative is sparse. For the sake of generalization, we consider in this article that the effects of dissipation may be present, and that is why the state equation is the whole Korteweg-de Vries-Burgers equation \cite{su2003korteweg}. Various difficulties arise in our existence theory specifically from a lack of dissipation but we suggest several ways out.

In this perspective, we follow the path introduced in \cite{clason2011duality,casas2012approximation}, and focus on the optimal control problem
\begin{multline}
\min_{u \in \M, y\in Y}J(y,u)=\frac{1}{2}\left(\norm{\chi_{\Omega_{o}}y - z_1}_{L^2(I\times \Omega_{o})}^2+\|\chi_{\Omega_{o}}y(T)-z_2\|_{L^2(\Omega_{o})}^2\right)\\
+ \alpha \norm{u}_{\mathcal{M}(\Omega_c, L^{2}(I))}
\label{cost}
\end{multline}
where $y\in Y$ is the solution of the Korteweg-de Vries-Burgers equation with Dirichlet and Neumann boundary conditions on $\Omega = (0,L)$ (the space $Y$ will be defined later on)
\begin{subequations}
\begin{numcases}{}
\partial_t y +\partial_x y + \partial_{xxx} y + y\partial_x y -\gamma \partial_{xx} y=  u \mbox{ in } I\times\Omega,\label{kdvcontrol1}\\
y(\cdot,0) = y(\cdot,L) = \partial_x y (\cdot,L) = 0\mbox{ in } I,\label{kdvcontrol2}\\
y(0,\cdot) = y_0(\cdot) \mbox{ in } \Omega\label{kdvcontrol3}.
\end{numcases}
\end{subequations}
The control acts on the control domain $\Omega_c\subseteq \Omega$. The state variable $y$ is tracked on the observation domain $\Omega_{o}\subseteq\Omega$. We consider the control cost term $\alpha > 0$ and the diffusion coefficient $\gamma \geq 0$. Denoting by $I=[0,T]$ the time horizon considered, the control variable $u$ lies in the space of Borel measures with values in $L^2(I)$ that we will denote $\M$.
%A crucial feature of our mathematical analysis is based on the fact that this space can be identified with the topological dual of $C_{0}(\Omega,L^2(I))$ \cite{clason2011duality,casas2012approximation}, i.e. the space of continuous functions with compact support in $\Omega$ and values in $L^2(I)$.
To give an insight, $\M$ contains functions of the type $u(t,x) = \sum_{i=1}^{N}{u_{i}(t)\delta_{x_{i}}(x)}$, with $u_i \in L^2(I)$ and $\delta_{x_i}$ ard Dirac delta functions located at the fixed points $x_i$. But we want to stress that it does not include moving Dirac delta functions. Those functions would rather be elements of $L^2(I,\mathcal{M}(\Omega))$, with $\M \subset L^2(I,\mathcal{M}(\Omega))$. This type of measure-valued control problems has already been studied in the case of linear elliptic and parabolic equation \cite{pieper2013priori,clason2011duality,casas2012approximation} and is known to promote directional sparsity of the control while being analytically tractable, unlike what an $L^1$ regularization term would provide.


To the authors knowledge, optimal control of the Korteweg-de Vries-Burgers equation is still an open problem, especially in a sparsity promoting framework. From a mathematical point of view, the challenge is twofold: on the one hand we shall prove well-posedness of the forward problem in the presence of an irregular source term while on the other hand sparse optimal control of a nonlinear dispersive partial differential equation is also a novel question. We point out that the quantities $y$, $x$, and $t$ can be rescaled to produce any desired coefficients for the terms of \eqref{kdvcontrol1} - \eqref{kdvcontrol3}. The choice we make here is convenient for the mathematical analysis of the problem.

%%% OUTLINE
This paper is organized as follows. Section~\ref{secwellposedness} and Section~\ref{wp2} deal with the well-posedness of the optimal control problem with measure-valued controls. This includes a study of the forward PDE problem with irregular data but also the question of the existence of an optimum. Section~\ref{secoptconditions} introduces the optimality conditions used to solve the problem. Afterwards, we expose in Section~\ref{secnum} the various numerical strategies we adopt for the simulation part and the optimization problem. We illustrate them with some numerical examples on the Korteweg-de Vries equation, from two points of view: an inverse problem and a control problem.

%%% Local Variables:
%%% mode: latex
%%% TeX-master: "kdv.tex"
%%% End:
