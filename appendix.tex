%!TEX root = kdv.tex
\appendix
\section{Well-posedness of the state equation}
\label{sec:appwp}

% \subsection{Semigroup of contractions}
% \label{sec:semigr-contr}
% \begin{proof}[Proof of Proposition~\ref{propsemigroup}]
%   The idea of the proof is not new and was first introduced in \cite{rosier1997exact}. We choose to provide it here as we are concerned with a slightly different operator (we add diffusion to the problem).
%   The idea is to prove that $A$ is maximally dissipative in order to
%   use a corollary of the Lumer-Philips theorem and conclude. We
%   already have that $A:\mathcal{D}(A) \mapsto L^{2}(0,L)$ has a dense
%   domain and it is easy to see that it is a closed operator. Let us
%   prove that it is dissipative. For any $w \in \mathcal{D}(A)$, \beal
%   <w,Aw>_{L^{2}(0,L)} &= \int_{0}^{L}{w(-w'''-w'+\gamma w'')dx}\\
%   & = -[ww'']_{0}^{L} + \int_{0}^{L}{w'w''dx} - [w^{2}]_{0}^{L} + \gamma [ww']_{0}^{L} - \gamma \int_{0}^{L}{w'^{2}dx}\\
%   & = -\frac{1}{2}w'(0)^{2} - \gamma \int_{0}^{L}{w'^{2}dx} \leq 0.
%   \eeal Hence A is dissipative. Denoting $A^{\ast}$ the adjoint
%   operator of $A$ satisfying \be A^{\ast}w = w''' + w' + \gamma w'',
%   \ee we also have \beal
%   <A^{\ast}w,w>_{L^{2}(0,L)} &= \int_{0}^{L}{w(w'''+w'+\gamma w'')dx}\\
%   & = -\frac{1}{2}w'(L)^{2} - \gamma \int_{0}^{L}{w'^{2}dx} \leq 0.
%   \eeal Therefore $A^{\ast}$ is also dissipative and A is maximally
%   dissipative. A corollary of the Lumer-Philips theorem theorem (see
%   \cite{pazy1983semigroups}, Chapter 1, Cor 4.4) permits to conclude
%   that A generates a strongly continuous semigroup of contractions.
% \end{proof}

\subsection{Linear estimates}
\label{sec:linear-estimates}
\begin{proof}[Proof linear estimate \eqref{linestimate_regular}]
  The proof is here largely inspired from \cite{rosier1997exact,glass2008some}. Let $y\in \mathcal C(\bar I,\mathcal D(A))~\cap~\mathcal C^1(\bar I,L^2(\Omega))$ be the classical solution of \eqref{kdvlinnonhom} for data $f\in \Hm1$ and $y_0\in \mathcal D(A)$. We multiply \eqref{kdvlinnonhom1} which holds in $L^2(\Omega)$ for a.e. $t\in I$ with $y$ and get
   \be
  \frac{1}{2}\frac{d}{dt}\int_{0}^{L}{y^{2}dx} + \abs{\partial_{x}y(t,0)}^{2} + \gamma \int_{0}^{L}{(\partial_{x} y)^{2}dx}=  \langle f,y\rangle_{H^{-1}(\Omega),H^{1}_{0}(\Omega)}.
  \ee
  Applying Cauchy-Schwarz followed by Young's inequality to the right-hand side
  leads to \be \frac{1}{2}\frac{d}{dt}\int_{0}^{L}{y^{2}dx} + \abs{\partial_{x} y(t,0)}^{2} + \gamma \int_{0}^{L}{(\partial_{x}y)^{2}dx}\leq \frac{1}{2}\norm{f}_{H^{-1}(\Omega)}^{2} + \frac{1}{2}\norm{y}_{H^{1}_{0}(\Omega)}^{2}
  \label{1linnhupperbound}.
  \ee
  We proceed in the same manner testing with $xy$
  \be
  \frac{1}{2}\frac{d}{dt}\int_{0}^{L}{xy^{2}dx}-\frac{1}{2}\int_{0}^{L}{y^{2}dx} +  \frac{3}{2}\int_{0}^{L}{(\partial_{x} y)^{2}dx} +\gamma
  \int_{0}^{L}{x(\partial_{x} y)^{2}dx}= \langle f,xy \rangle_{H^{1}_{0}(\Omega)}.
  \label{2linnhupperbound}
  \ee
  The right-hand side is treated again thanks to Cauchy-Schwarz
  and Young's inequalities
  \beal
  \langle f,xy\rangle_{H^{-1}(\Omega),H^{1}_{0}(\Omega)} &\leq \norm{q}_{H^{-1}(\Omega)}\norm{xy}_{H^{1}_{0}(\Omega)}\\
  & \leq \norm{f}_{H^{-1}(\Omega)}\norm{y + x\partial_{x}y}_{L^{2}(\Omega)}\\
  % & \leq \norm{q}_{H^{-1}(\Omega)} \left( \norm{y}_{L^{2}(\Omega)} + \norm{x\partial_{x}y}_{L^{2}(\Omega)} \right)\\
  & \leq \norm{f}_{H^{-1}(\Omega)} \left( \norm{y}_{L^{2}(\Omega)} + L\norm{\partial_{x}y}_{L^{2}(\Omega)} \right)\\
  &\leq \frac{1}{2}\norm{f}_{H^{-1}(\Omega)}^{2} + \frac{1}{2}\norm{y}_{L^{2}(\Omega)}^{2} + \frac{L^{2}}{2}\norm{f}_{H^{-1}(\Omega)}^{2} + \frac{L}{2L}\norm{\partial_{x}y}_{L^{2}(\Omega)}^{2}\\
  &\leq \frac{1+L^{2}}{2}\norm{f}_{H^{-1}(\Omega)}^{2} +  \frac{1}{2}\norm{y}_{L^{2}(\Omega)}^{2} +  \frac{1}{2}\norm{\partial_{x}y}_{L^{2}(\Omega)}^{2}
  \label{qupperbound}
  \eeal
  Adding \eqref{1linnhupperbound}, \eqref{2linnhupperbound}(with the upper bound \eqref{qupperbound}) and omitting some non-negative terms on the left-hand side yields \be
  \frac{1}{2}\frac{d}{dt}\int_{0}^{L}{(1+x)y^{2}dx} +(\frac{1}{2}+\gamma)\int_{0}^{L}{(\partial_{x}y)^{2}} \leq \left(   1 +\frac{L^{2}}{2}\right)\norm{f}_{H^{-1}(\Omega)}^{2} + \int_{0}^{L}{y^{2}dx}.
  \ee
   To facilitate the next computations, we add a non-negative term on the right and multiply by two
   \be
 \frac{d}{dt}\int_{0}^{L}{(1+x)y^{2}dx} + (1+2\gamma)\int_{0}^{L}{(\partial_{x}y)^{2}} \leq \left(    2 +L^{2}\right)\norm{f}_{H^{-1}(\Omega)}^{2} +  2\int_{0}^{L}{(1+x)y^{2}dx}.
  \ee
  Then we can follow a Gronwall strategy, multiplying by $e^{-t}$
  \beal
  \frac{d}{dt}\left(e^{-2t}\int_{0}^{L}{(1+x)y^{2}dx}\right) + e^{-2t}\left(1+2\gamma\right)\int_{0}^{L}{(\partial_{x}y)^{2}} \leq
  e^{-2t}\left( 2+L^{2}\right)\norm{f}_{H^{-1}(\Omega)}^{2}
  \eeal
  After integration between $0$ and $t$  and multiplication by $e^{2t}$we have
  \beal
  \int_{0}^{L}{y^{2}(t)dx} + \left( 1+2\gamma \right)\int_{0}^{t}{\int_{0}^{L}{(\partial_{x} y)^{2}dxdt}} \leq e^{2t}\left(2+L^{2}\right)&\int_{0}^{t}{\norm{f}_{H^{-1}(\Omega)}^{2}}\\
  & + e^{2t}\int_{0}^{L}{(1+L)y_{0}^{2}dx},
  \eeal
  that we transform into
  \beal
  \int_{0}^{L}{y^{2}(t)dx} + \left( 1+2\gamma \right)\int_{0}^{t}{\int_{0}^{L}{(\partial_{x} y)^{2}dxdt}} \leq e^{2T}\left(2+L^{2}\right)&\norm{f}_{L^{2}(I,H^{-1}(\Omega))}^{2} \\
  &+ e^{2T}(1+L)\norm{y_{0}}_{L^{2}(\Omega)}.
  \eeal
  This yields
  \[\|y\|_{\mathcal C(\bar I,L^2(\Omega))}+\|y\|_{L^2(I,H^1_0(\Omega))}\leq c(\|f\|_{\Hm1}+\|y_0\|_{L^2(\Omega)}).\]



  %Because
%  it is a one-dimensional problem, there holds
%$$\M \hookrightarrow L^{2}(I, \mathcal{M}(\Omega)) \hookrightarrow L^{2}(I, H^{-1}(\Omega)).$$
%Therefore we obtain estimate \eqref{linestimate} for any $y_{0} \in
%\mathcal{D}(A)$ and $q \in C_{0}([0,T], \mathcal{D}(A))$. Let us
%conclude by a density argument. We consider two sequences
%$$ y_{0}^{n} \in \mathcal{D}(A) \xrightarrow[n\rightarrow+\infty]{}y_{0} \in L^{2}(\Omega),\quad
%q^{n} \in  C_{0}([0,T], \mathcal{D}(A)) \xrightarrow[n\rightarrow+\infty]{} q \in L^{2}(I,H^{-1}(\Omega)).$$
%We associate to any pair $\left(y_{0}^{n}, q^{n}\right)$ the sequence
%of solutions $(y^{n})$. Due to the linearity of the considered
%equation, the estimate \eqref{linestimate} implies that $(y^{n})$ is a
%Cauchy sequence in $\mathcal{B}$ that converges towards some $y \in
%\mathcal{B}$. Besides, uniqueness follows from semigroup theory and
%uniqueness of the limit. This concludes the proof.
\end{proof}

% \subsection{Weak formulation}
% \label{sec:weak-formulation}
% \begin{proof}[Proof for Remark~\ref{rmkweakform}]
%   In the proof of Proposition~\ref{propnonhomo}, the solution $y$ to \eqref{kdvlinnonhom} is defined as the limit in $\mathcal{B}$ of the smooth solution $y^n$ of \eqref{kdvlinnonhom} obtained for smooth initial conditions and right-hand side
% \beal
% & y_0^n \in \mathcal{D}(A)\underset{n\to +\infty}{\longrightarrow} y_0 \in L^2(\Omega),\quad
% & q_n \in C(0,T,\mathcal{D}(A))\underset{n\to +\infty}{\longrightarrow}q \in \M.
% \eeal
% This solution $y^n$, because of its regularity, clearly satisfies \eqref{kdvlinnonhom} in a classical sense, a fortiori in the sense of \eqref{weakform}, obtained after successive integrations by part
% \be
% -(y^n,\partial_t \varphi)_I + (\partial_x y^n, \varphi)_I + (\partial_x y^n, \partial_{xx}\varphi)_I + \gamma (\partial_x y^n, \partial_x \varphi)_I = <q^n,\varphi>, \quad \forall \varphi \in \mathcal{V}.
% \ee
% Then, for each term, using on the one hand the embeddings $\mathcal{V}\hookrightarrow L^2(I,H^2\cap H^1_0)\hookrightarrow L^2(I, L^2(\Omega))$, and on the other hand the regularity of $y$, one can write
% \beal
% \abs{(y^n-y,\partial_t \varphi)_I}\leq \norm{y^n - y}_{L^2(I\times\Omega)} \norm{\partial_t \varphi}_{L^2(I\times\Omega)} &\leq \norm{y^n - y}_{L^2(I,H^1_0(\Omega))} \norm{ \varphi}_{H^1(0,T,L^2(\Omega))}\\
% &\leq \norm{y^n - y}_{\mathcal{B}} \norm{\varphi}_{\mathcal{V}}
% \eeal
%
% \beal
% \abs{(\partial_x (y^n-y),\varphi)_I}\leq \norm{\partial_x (y^n - y)}_{L^2(I\times\Omega)} \norm{\varphi}_{L^2(I\times\Omega)} &\leq \norm{y^n - y}_{L^2(I,H^1_0(\Omega))} \norm{\varphi}_{\mathcal{V}} \\
% &\leq \norm{y^n - y}_{\mathcal{B}} \norm{\varphi}_{\mathcal{V}}
% \eeal
%
% \beal
% \abs{(\partial_x( y^n-y),\partial_{xx} \varphi)_I}\leq \norm{\partial_x (y^n - y)}_{L^2(I\times\Omega)} \norm{\partial_{xx} \varphi}_{L^2(I\times\Omega)} &\leq \norm{y^n - y}_{L^2(I, H^1_0(\Omega))}\norm{\varphi}_{L^2(I, H^2 \cap H^1_0(\Omega))}\\
% & \leq \norm{y^n - y}_{\mathcal{B}} \norm{\varphi}_{\mathcal{V}}
% \eeal
%
% \beal
% \abs{(\partial_x (y^n-y),\partial_x \varphi)_I}\leq \norm{\partial_x( y^n-y)}_{L^2(I\times\Omega)} \norm{\partial_x \varphi}_{L^2(I\times\Omega)} &\leq  \norm{y^n - y}_{L^2(I, H^1_0(\Omega))} \norm{\varphi}_{L^2(I, H^1_0(\Omega))}\\
% &\leq \norm{y^n - y}_{\mathcal{B}} \norm{\varphi}_{\mathcal{V}}
% \eeal
% Recall that $\norm{y^n - y}_{\mathcal{B}} \norm{\varphi}_{\mathcal{V}} \underset{n\to +\infty}{\longrightarrow} 0$. Moreover, since $q^n$ tends to $q$ in $L^2(I,H^{-1}(\Omega))$, one directly has
% \be
% <q^n,\varphi> \underset{n\to +\infty}{\longrightarrow} <q,\varphi>.
% \ee
% As a consequence, $y$ satisfies
% \be
% -(y,\partial_t \varphi)_I + (\partial_x y, \varphi)_I + (\partial_x y, \partial_{xx}\varphi)_I + \gamma (\partial_x y, \partial_x \varphi)_I = <q,\varphi>, \quad \forall \varphi \in \mathcal{V}.
% \ee
% \end{proof}



\subsection{Nonlinear state equation}
\label{sec:nonl-state-equat}
\begin{proof}[Proof of Lemma~\ref{lemyyx2}] The technique is inspired from (\cite{faminskii2010initial}, Proof of Theorem 2.8). Let us consider $y \in \B$ and $z \in \B$. It holds $\mathcal B\hookrightarrow L^4(I\times \Omega)$ and therefore we can estimate using $\|y\|_{L^\infty(\Omega)}^2\leq c\|y\|_{L^2(\Omega)}\|y\|_{H^1_0(\Omega)}$
\beal
\norm{y\partial_x y - z\partial_x z}_{\Hm1}  & = \frac 1 2\left(\int_0^T{\left(\ \sup_{\norm{\varphi}_{H^1_0(\Omega) = 1}}(y^2 -  z^2,\partial_x\varphi)_{L^2(\Omega)}\right)^2}\mathrm dt\right)^{1/2} \\
& \leq \frac{1}{2}\norm{z-y}_{L^4(I\times \Omega)}\norm{z+y}_{L^4(I\times \Omega)} \\
& \leq\frac{1}{2} \norm{z-y}_{C(0,T,L^2(\Omega))}^{1/2}\left( \int_0^T{\norm{z-y}_{L^{\infty}(\Omega)}^2}\right)^{1/4}\\
&\quad \quad\cdot\norm{z+y}_{C(0,T,L^2(\Omega))}^{1/2}\left( \int_0^T{\norm{z+y}_{L^{\infty}(\Omega)}^2}\right)^{1/4}\\
& \leq c\, \norm{z-y}_{C(0,T,L^2(\Omega))}^{1/2}\norm{z+y}_{C(0,T,L^2(\Omega))}^{1/2}\\
& \quad \quad\cdot \left(\int_0^T{ \norm{z-y}_{L^2(\Omega)}\norm{z-y}_{H^1_0(\Omega)}}\right)^{1/4}\left(\int_0^T{ \norm{z+y}_{L^2(\Omega)}\norm{z+y}_{H^1_0(\Omega)}}\right)^{1/4}\\
& \leq c \,T^{1/4}\norm{z-y}_{C(0,T,L^2(\Omega))}^{3/4}\norm{z-y}_{L^2(0,T,H^1_0(\Omega))}^{1/4}\\
& \quad \quad \cdot\norm{z+y}_{C(0,T,L^2(\Omega))}^{3/4}\norm{z+y}_{L^2(0,T,H^1_0(\Omega))}^{1/4} \\
&\leq c \,T^{1/4}\norm{y-z}_{\B}\norm{y+z}_{\B}
\eeal
\end{proof}

%\begin{proof}[Proof of Proposition~\ref{localposedness}, uniqueness]
%We now prove the uniqueness of the (weak) solution of the nonlinear KdV equation \eqref{kdvcontrol1} - \eqref{kdvcontrol3}. Let us first consider two solutions of the same Cauchy problem $y$ and $z$ defined on $[0,T^{\ast}]\times\Omega$. Then $u = y-z$ is a solution of
%\bealn
%&\partial_t u +\partial_x u + \partial_{xxx} u -\gamma \partial_{xx} u= - y\partial_x u - u\partial_x z \mbox{ in }   I\times\Omega,\\
%&u(.,0) = u(.,L) = \partial_x u (.,L) = 0 \mbox{ on } I\times\Gamma,\\
%&u(0,.) = 0 \mbox{ in } \Omega,
%\label{kdvnonlin1}
%\eealn
%Multiplying by $2xu$ and integrating in $x$ (as proposed in \cite{rosier1997exact,coron2003exact}) leads to
%\be
%\int_{0}^{L}{2xu\left( \partial_t u +\partial_x u + \partial_{xxx} u -\gamma \partial_{xx} u+ y\partial_x u + u\partial_x z\right)dx} = 0,
%\ee
%which also writes
%\be
%\frac{d}{dt}\int_{0}^{L}{xu^2dx} + 3\int_0^L{(\partial_x u)^2dx} +  2\gamma\int_0^L{x(\partial_x u)^2dx} = \int_0^L{u^2dx} - 2\int_0^L{x y u \partial_x udx} + 2\int_0^L{zu^2 dx}+4\int_0^L{x z u \partial_x u dx}
%\ee
%Then, we follow \cite{coron2003exact} to upperbound every term on the right hand side. Thanks to the continuous embedding of $H^1_0(\Omega)$ into $C^0(\Omega)$, there exists a positive constant $C$ such that
%\be
%2\megaabs{\int_0^L{xy u \partial_x u dx}} \leq C_1 \norm{\partial_x y}_{L^2(\Omega)}\int_0^L{\abs{x u \partial_x u}dx}
%\label{eq1}
%\ee
%Using Cauchy-Schwarz and Young's inequalities leads to
%\be
%2\megaabs{\int_0^L{xy u \partial_x u dx}} \leq \frac{1}{2}\int_0^L{\left(\partial_x u\right)^2dx} + \frac{C_1^2}{2}\norm{\partial_x y}_{L^2(\Omega)}^2 L\int_0^L{x u^2 dx}.
%\label{eq2}
%\ee
%And the same process is applied to
%\be
%4\megaabs{\int_0^L{x z u \partial_x u dx}} \leq \frac{1}{2}\int_0^L{\left(\partial_x u\right)^2dx} + 2 C_1^2\norm{\partial_x z}_{L^2(\Omega)}^2 L\int_0^L{x u^2 dx}.
%\label{eq3}
%\ee
%Recalling from \cite{coron2003exact} the lemma
%\begin{lem}
%For every $\phi \in H^1_0(0,L)$ with $\phi(0) = 0$ and every $a \in [0,L]$,
%\be
%\int_0^L{\phi^2dx} \leq \frac{a^2}{2}\int_0^L{\left(\partial_x \phi \right)^2 dx} + \frac{1}{a}\int_0^L{x\phi^2 dx},
%\ee
%\label{lem1}
%\end{lem}
%\noindent one can prove that there exists $C_{2}$ such that
%\be
%\int_0^L{u^{2}dx} \leq \frac{1}{2}\int_{0}^{L}{\left( \partial_{x}u\right)^{2}dx} + C_{2}\int_{0}^{L}{xu^{2}dx}.
%\label{eq4}
%\ee
%Finally, using the same justification as \eqref{eq1}, there exists $C_{3}$ such that
%\be
%2\megaabs{\int_0^L{zu^{2} dx}} \leq C_{3}\norm{z_{x}}_{L^{2}(0,L)}\int_{0}^{L}{u^{2}dx},
%\ee
%Combined with \eqref{eq4}, this latter inequality rewrites, for a constant $C_{4}$
%\be
%2\int_0^L{zu^{2} dx} \leq \frac{1}{2}\int_{0}^{L}{\left( \partial_{x} u\right)^{2}dx} + C_{4} \left( 1 + \norm{z_{x}}_{L^{2}(0,L)}^{3/2}\right) \int_{0}^{L}{xu^{2}dx}
%\label{eq5}
%\ee
%Now, by \eqref{eq2}, \eqref{eq3}, \eqref{eq4}, \eqref{eq5}, we have
%\beal
%\frac{d}{dt}\int_{0}^{L}{xu^2dx} + \int_0^L{(\partial_x u)^2dx} \leq C_5 \left( 1 + \norm{\partial_x y}_{L^2(\Omega)}^2 + \norm{\partial_x z}_{L^2(\Omega)}^2 \right)\int_0^L{ xu^2 dx}
%\label{eq6}
%\eeal
%for a given constant $C_5$.
%In particular, applying Gronwall lemma to
%\be
%\frac{d}{dt}\int_{0}^{L}{xu^2dx} \leq C_5 \left( 1 + \norm{\partial_x y}_{L^2(\Omega)}^2 + \norm{\partial_x z}_{L^2(\Omega)}^2 \right)\int_0^L{ xu^2 dx}
%\ee
%leads to
%\be
%\int_{0}^{L}{xu^2dx} \leq \left[\int_{0}^{L}{xu_0^2dx}\right]\, e^{\displaystyle \int_0^s{C_5 \left( 1 + \norm{\partial_x y}_{L^2(\Omega)}^2 + \norm{\partial_x z}_{L^2(\Omega)}^2 \right)ds}} = 0,
%\label{eqend}
%\ee
%since in our case $u_0 = 0$, \eqref{eqend} leads to $u = 0$ in $C^0(I,L^2(\Omega))$. Moreover, using again \eqref{eq6}
%we have
%\be
%\int_0^{T^{\ast}}{\int_0^L{(\partial_x u)^2dx}} \leq \int_0^{T^{\ast}}{C_5 \left( 1 + \norm{\partial_x y}_{L^2(\Omega)}^2 + \norm{\partial_x z}_{L^2(\Omega)}^2 \right)\int_0^L{ xu^2 dx}} = 0,
%\ee
%which leads also to $u = 0$ in $L^2(I, H^1_0(\Omega))$. Unicity of the solution to the nonlinear KdV system is thus proved.
%\end{proof}

\subsection{The tangent equation}\label{appendixtangent}
Next we analyze the well posedness of the tangent equation
\begin{subequations}
 \begin{numcases}{}
\partial_t \delta y +\partial_x \delta y + \partial_{xxx} \delta y - \gamma \delta \partial_{xx} y  + \partial_x(y \delta y)=  \delta u \mbox{ in } I\times\Omega,\label{linkdv1}\\
\delta y(.,0) = \delta y(.,L) = \partial_x \delta y (.,L) = 0 \mbox{ in } I,\label{linkdv2}\\
\delta y(0,x) = \delta y_0 \mbox{ in } \Omega.\label{linkdv3}
 \end{numcases}
\end{subequations}
\begin{Def}
Let $(\delta u,\delta y_0)\in \Hm1\times \mathcal B$ and $y\in \mathcal B$. A function $\delta y\in \mathcal B$ is called a solution of \eqref{linkdv1}-\eqref{linkdv3} if it satisfies the fixed point equation
\[
\delta y=\mathcal L(\delta u-\partial_x(y\delta y),\delta y_0)
\]
where $\mathcal L$ is the solution operator from Remark \ref{rmklinearoperator}.
\end{Def}
\begin{prop}\label{prop:tangent}
 Let $(\delta u,\delta y_0) \in \Hm1\times L^2(\Omega)$ and $y\in \mathcal B$. Then, there exists a unique solution $\delta y \in \mathcal{B}$ of \eqref{linkdv1}-\eqref{linkdv3}. Furthermore, there exists a constant $\widetilde C(T,L,\norm{y}_{\mathcal{B}})$ such that the following estimate holds
 \be\label{estimatetangent}
 \norm{\delta y}_{\mathcal B}+\|\partial_t \delta y\|_{L^2(I,\mathcal V^*)} \leq \widetilde C \left( \norm{\delta y_0}_{L^2(\Omega)} + \norm{\delta u}_{\Hm1}\right).
 \ee
\end{prop}
\begin{proof}[Proof of Proposition~\ref{prop:tangent}]
We define the linear mapping
\[
\Psi_{\delta u, \delta y_0,y}\colon \mathcal B_{\theta}\rightarrow \mathcal B_{\theta},~~\Psi_{\delta u, \delta y_0,y}(\delta y) = \mathcal{L}(\delta u - \partial_x(y\delta y),\delta y_0)
\]
with $\mathcal B_{\theta}$ defined as in \eqref{btheta} and \eqref{normbtheta} and $\mathcal{L}$ being the linear \KdV operator described in Remark~\ref{rmklinearoperator}. Our goal is to show that under some constraints on $\theta$, $\Psi_{\delta u, \delta y_0,y}$ is a contraction mapping, such that the Banach fixed point theorem can be applied. First we estimate $\partial_x(y \delta y)$ in the $L^2([0,\theta],H^{-1}(\Omega))$-norm
\beal\label{estimate_variable_coefficient}
\|\partial_x(y\delta y)\|_{L^2([0,\theta],H^{-1}(\Omega))} &\leq \left(\int_{0}^{\theta}\left(\underset{\|v\|_{H^1_0(\Omega)}\leq 1}{\operatorname{sup}}\langle\partial_x(y\delta y)(t),v\rangle_{H^{-1}(\Omega),H^1_0(\Omega)}\right)^2~\mathrm dt\right)^{\frac{1}{2}}\\
&\leq\left(\int_{0}^{\theta}\|\delta y(t)\|_{L^2(\Omega)}^2\|y(t)\|_{L^\infty(\Omega)}^2~\mathrm dt\right)^{\frac{1}{2}}\\
&\leq c\,\|\delta y\|_{\mathcal C([0,\theta],L^2(\Omega))}\left(\int_{0}^{\theta}\|y(t)\|_{H^1_0(\Omega)}\|y(t)\|_{L^2(\Omega)}~\mathrm dt\right)^{\frac{1}{2}}\\
&\leq c\,\theta\,\|\delta y\|_{\mathcal C([0,\theta],L^2(\Omega))}\left(\int_{0}^{\theta}\|y(t)\|_{H^1_0(\Omega)}^2\|y(t)\|_{L^2(\Omega)}^2~\mathrm dt\right)^{\frac{1}{2}}\\
&\leq c\,\theta\,\|\delta y\|_{\mathcal C([0,\theta],L^2(\Omega))}\|y\|_{\mathcal C([0,\theta],L^2(\Omega))}\|y\|_{L^2([0,\theta],H^1_0(\Omega))}\\
&\leq c\,\theta\,\|y\|_{\mathcal B_{\theta}}^2\|\delta y\|_{\mathcal B_{\theta}}
\eeal
Therefore we can estimate
\beal
\|\Psi(\delta y)_{\delta u, \delta y_0,y}\|_{\mathcal B_{\theta}} & \leq \widetilde{C}\left(\|\delta y_0\|_{L^2(\Omega)}+\|\delta u\|_{L^2(I,H^{-1}(\Omega))}+\|\partial_x(y\delta y)\|_{L^2([0,\theta],H^{-1}(\Omega))}\right)\\
&\leq \widetilde{C}\left(\|\delta y_0\|_{L^2(\Omega)}+\|\delta u\|_{L^2(I,H^{-1}(\Omega))}\right) + C_2\theta\|y\|_{\mathcal B_{\theta}}^2\|\delta y\|_{\mathcal B_{\theta}}
\eeal
and
\be
\|\Psi_{\delta u, \delta y_0,y}(\delta y_1)-\Psi_{\delta u, \delta y_0,y}(\delta y_2)\|_{\mathcal B_{\theta}}\leq C_2\theta\|y\|_{\mathcal B_{\theta}}^2\|\delta y_1-\delta y_2\|_{\mathcal B_{\theta}}.
\ee
Now we set $r=2 \widetilde{C}\left(\|\delta y_0\|_{L^2(\Omega)}+\|\delta u\|_{L^2(I,H^{-1}(\Omega))}\right)$ and introduce the ball
\[
B=\{\delta y \in \mathcal B_{\theta}\colon \|\delta y\|_{\mathcal B_{\theta}}\leq r\}.
\]
Next we choose $\theta$ small enough such that
\[
C_2\theta\|y\|_{\mathcal B}^2\leq\frac{1}{3}
\]
holds. Then the following inequalities hold
\[
\|\Psi_{\delta u, \delta y_0,y}(\delta y)\|_{\mathcal B_{\theta}}\leq \frac{2}{3}r,~~\|\Psi_{\delta u, \delta y_0,y}(\delta y_1) - \Psi_{\delta u, \delta y_0,y}(\delta y_2)\|_{\mathcal B_{\theta}}\leq\frac{2}{3}\|\delta y_1-\delta y_2\|_{\mathcal B_{\theta}}.
\]
which implies that $\Psi_{\delta u, \delta y_0,y}$ is a contraction mapping on $B$. So we can apply the Banach fixed point theorem which guarantees the existence of a unique fixed point $\delta y$ of $\Psi_{\delta u, \delta y_0,y}$ which is a solution of \eqref{linkdv1}-\eqref{linkdv3} in $(0,\theta)$ with initial value $\delta y_0$. Since $\theta$ is independent of $\delta y_0$ we can apply this strategy successively starting at $t=0$ with $\delta y_0$ and using $\delta y(k\theta)$ as initial points for $k=1,2,3,\ldots,N$ until $T$ is reached. The concatenation of all $\delta y(k\theta)$ for $k=1,2,3,\ldots,N$ is a solution of \eqref{linkdv1} - \eqref{linkdv3}. Existence of a unique solution is thus proven. Concerning the estimates \eqref{estimatetangent}, the proof is very similar to the non-variable coefficients case (see Appendix~\ref{sec:linear-estimates}). We just mention the main differences.  In the case of  a smooth solution $\delta y \in \mathcal C(\bar I,\mathcal D(A))\cap \,\mathcal C^1(I,L^2(\Omega))$ we multiply \eqref{linkdv1} by $\delta y$ and estimate the term concerning $y$ in the following way
\beal
\megaabs{\int_0^L{\delta y \partial_x \left( y \delta y \right)}~\mathrm dx} = \megaabs{-\int_0^L{y \delta y \partial_x \delta y}~\mathrm dx} &\leq \frac{1}{2\varepsilon}\int_0^L{y^2 \delta y^2}~\mathrm dx + \frac{\varepsilon}{2}\|\delta y\|_{H^1_0(\Omega)}^2\\
&\leq \frac{1}{2\varepsilon}\norm{y}_{L^{\infty}(\Omega)}^2 \|\delta y\|^2_{L^2(\Omega)} + \frac{\varepsilon}{2}\|\delta y\|_{H^1_0(\Omega)}^2
\eeal
for any $\varepsilon>0$.
In the same manner, multiplying \eqref{linkdv1} by $x\delta y$ leads to
\beal
\megaabs{\int_0^L{x\delta y \partial_x(y \delta y)}}~\mathrm dx &= \megaabs{- \int_0^Ly \delta y^2~\mathrm dx - \int_0^Lxy\delta y
\partial_x \delta y~\mathrm dx}\\
&\leq \left(\norm{y}_{L^{\infty}(\Omega)}+\frac{L^2}{2\varepsilon}\norm{y}_{L^{\infty}(\Omega)}^2\right)\|\delta y\|_{L^2(\Omega)}^2 +\frac{\varepsilon}{2}\|\delta y\|_{H^1_0(\Omega)}^2 \\
\eeal
for any $\varepsilon>0$. Based on these estimates the a priori estimate \eqref{estimatetangent} can be shown. The estimate for $\|\partial_t y\|_{L^2(I,\mathcal V^*)}$ can be shown based on \eqref{estimate_variable_coefficient} and \eqref{estimatetangent}.
\end{proof}

\subsection{The adjoint equation}
\label{appendixadjoint}
Next we study the following equation
\besn
-\partial_t p -\partial_x p - \partial_{xxx} p - \gamma \partial_{xx} p  - y\partial_x p =  \phi \mbox{ in } I\times\Omega,\label{adjoint1}\\
p(.,0) = p(.,L) = \partial_x p (.,0) = 0 \mbox{ on } I,\label{adjoint2}\\
p(T) = p_{T} \mbox{ in } \Omega.\label{adjoint3}
\eesn
for any $y\in \mathcal B$.
\begin{Def}
A function $p\in \mathcal B$ is called a solution of \eqref{adjoint1}-\eqref{adjoint3} if it solves the fixed point equation
\[
p(t)=W^*(t)p_T+\int_t^TW^*(s-t)(\phi(s)+y(s)\partial_x p(s))~\mathrm ds.
\]
\end{Def}
\begin{prop}
Let $(\phi,p_T)\in L^1(I,L^2(\Omega))\times L^2(\Omega)$. Then the equation \eqref{adjoint1}-\eqref{adjoint3} has a unique solution $p\in \mathcal B$. Furthermore there exists a constant $c\,(\|y\|_{\mathcal B})>0$ such that
\begin{equation}\label{apriori_adjoint}
\|p\|_{\mathcal B}\leq c\,(\|p_T\|_{L^2(\Omega)}+\|\phi\|_{L^1(I,L^2(\Omega))})
\end{equation}
holds.
\end{prop}
\begin{proof}
The proof uses similar arguments as the proof of Proposition \ref{prop:tangent}. In particular it is based on the Banach fixed theorem and Lemma \eqref{lemadjoint}. The estimate \eqref{apriori_adjoint} follows from
\begin{align*}
\left|\int_0^Ly\partial_xpp~\mathrm dx\right|=\left|-\int_0^Lp^2\partial_xy~\mathrm dx\right|&\leq c\,\|p\|_{H^1_0(\Omega)}\|p\|_{L^2(\Omega)}\|y\|_{H^1_0(\Omega)}\\
&\leq \frac{c}{2\varepsilon}\|p\|_{L^2(\Omega)}^2\|y\|_{H^1_0(\Omega)}^2+\frac{\varepsilon}{2}\|p\|_{H^1_0(\Omega)}^2
\end{align*}
for any $\varepsilon>0$ and
\begin{align*}
\left|\int_0^Ly\partial_xp(L-x)p~\mathrm dx\right|=\left|-\int_0^L(L-x)p^2\partial_xy-yp^2~\mathrm dx\right|&\leq c\,\|p\|_{H^1_0(\Omega)}\|p\|_{L^2(\Omega)}\|y\|_{H^1_0(\Omega)}\\
&\leq \frac{c}{2\varepsilon}\|p\|_{L^2(\Omega)}^2\|y\|_{H^1_0(\Omega)}^2+\frac{\varepsilon}{2}\|p\|_{H^1_0(\Omega)}^2.
\end{align*}
for any $\varepsilon>0$.
\end{proof}
%First we transform \eqref{adjoint1}-\eqref{adjoint3} into an initial value
%problem by the change of variable $\tilde{t} = T - t$, introducing
%$\tilde{p}$ such that $\tilde{p}(\tilde{t}) = p(t)$. Our problem then
%becomes
%\besn
%\partial_{\tilde{t}} \tilde{p}= \partial_x \tilde{p} + \partial_{xxx} \tilde{p} + \gamma \partial_{xx} \tilde{p}  + y\partial_x \tilde{p} - \tilde{f}\mbox{ in } I\times\Omega,\label{adjointrev1}\\
%\tilde{p}(.,0) = \tilde{p}(.,L) = \partial_x \tilde{p} (.,0) = 0 \mbox{ on } I,\label{adjointrev2}\\
% \tilde{p}(0,x) = p_{T}(x) \mbox{ on } \Omega.\label{adjointrev3}
%\eesn
%We denote by $A^*$ the linear differential operator \be A^*w =
%w''' + w' + \gamma w'', \ee on the dense domain
%$\mathcal{D}(A^*)\subset L^{2}(0,L)$ defined by \be \mathcal{D}(A^*) =
%\left\{w\in H^{3}(0,L) \mbox{ s.t. } w(0) = w(L) = w'(0) = 0\right\}.
%\ee Thus, \eqref{adjointrev1} - \eqref{adjointrev3} can be written as the initial value problem
%of an abstract evolution equation in the space $L^{2}(0,L)$
%\besn
%\frac{d}{d\tilde{t}}\tilde{p}(t)=A^*\tilde{p} +\tilde{y}\partial_x \tilde{p} + \tilde{f},\label{evolutionlinear1}\\
%\tilde{p}(0,x) = p_{T}(x).
%\label{evolutionlinear2}
%\eesn
%The following result holds
%\begin{prop}
%  $A^*$ generates a strongly continuous semigroup of contractions on
%  $L^{2}(0,L)$ that we denote $W^*_{0}(t)$.
%  \label{propsemigroupadjoint}
%\end{prop}
%
%\begin{proof}[Proof of Prop.~\ref{propsemigroupadjoint}]
%  We clearly have that $A^*$ is a closed operator and we have already
%  proven that $A^*$ is a dissipative operator. So is its adjoint
%  $A$. The Lumer-Philips theorem allows us to conclude.
%\end{proof}


%\subsubsection{The nonhomogeneous problem}
%First of all, let us point out that it is equivalent to study $p(t)$
%or $\tilde{p}(\tilde{t})$. Therefore we will drop the tilda
%sign from now on. The existence of this semigroup of contractions
%allows us, like in the case of the state equation, to study the
%nonhomogeneous problem
%\besn
%\partial_t p - A^*p = f \mbox{ in } I\times\Omega\label{nonhomoadjoint1}\\
%\tilde{p}(.,0) = \tilde{p}(.,L) = \partial_x \tilde{p} (.,0) = 0 \mbox{ on } I,\label{nonhomoadjoint2}\\
%p(0,.) = p_0(.)
%\label{nonhomoadjoint3}
%\eesn
%where $f\in \mathcal{B}$. Since $\mathcal{B}\subset
%L^1(0,T,L^2(\Omega))$, semigroup theory tells us that
%\eqref{nonhomoadjoint1} - \eqref{nonhomoadjoint3} admits a unique mild solution $p\in
%C(0,T,L^2(\Omega))$ which can be written with the Duhamel's formula
%\be p(t,x) = W_0^*(t)p_0(x) + \int_0^t{W_0^*(t-s)f(s)ds}
%\label{duhameladjoint}
%\ee where $W_0^*$ is the semigroup introduced in
%Proposition~\ref{propsemigroupadjoint}. Once again, we are able to
%prove
%\begin{prop}\label{existencenonhomoadjoint}
%  Let $f \in L^1(0,T,L^2(\Omega)$, $p_0\in L^2(0,L)$. Then, there
%  exists a unique solution $p \in \mathcal{B}$ satisfying \eqref{nonhomoadjoint1} - \eqref{nonhomoadjoint3}. This solution can be written according to Duhamel's formula
%  \eqref{duhameladjoint}. Furthermore there exists a constant $C(T,L)
%  > 0$ such that the following estimate holds \be \norm{p}_{C^0(I,
%    L^2(\Omega))} + \norm{p}_{L^2(I, H^1_0(\Omega))} \leq C(T,L)
%  \left(\norm{p_{0}}_{L^{2}(\Omega)} + \norm{f}_{L^1(0,T,L^2(\Omega))}
%  \right)
%  \label{linestimateadjoint}
%  \ee
%\end{prop}
%\begin{proof}[Proof of Proposition~\ref{existencenonhomoadjoint}]
%  The first part of the estimate follows from
%  \eqref{duhameladjoint}. This yields indeed \be
%  \norm{p}_{L^2(\Omega)} \leq \norm{p_0}_{L^2(\Omega)} +
%  \int_0^t{\norm{W_0^*(t-s)f(s,.)}_{L^2(\Omega)}ds}.  \ee Since \be
%  \norm{1_{[0,t]}W_0^*(t-s)f(s,.)}_{L^2(\Omega)}\leq
%  \norm{f(s,.)}_{L^2(\Omega)} \quad \in L^1(0,T), \ee it follows from
%  Lebesgue's theorem that the mild solution $p$ belongs to
%  $C(0,T,L^2(\Omega))$ and satisfies the estimate \be
%  \norm{p(t,.)}_{L^2(\Omega)} \leq \norm{p_0}_{L^2(\Omega)} +
%  \norm{f}_{L^1(0,T,L^2(\Omega))}.  \label{c0estimateadjoint}\ee To get the other part of the
%  estimate, let us consider first $p_{0} \in \mathcal{D}(A)$ and $f
%  \in C_{0}([0,T], \mathcal{D}(A))$ and proceed with the method of
%  multipliers like in the homogeneous case. We multiply
%  \eqref{nonhomoadjoint1} by $(L-x)p$ and integrate in space and time. The computations being highly identical to what was done for the state equation or the tangent equation, we do not detail it here.
%%   \be
%%   \frac{1}{2}\frac{d}{dt}\int_0^L{(L-x)p^2 dx} -
%%   \frac{1}{2}\int_0^L{p^2dx} + \frac{3}{2}\int_0^L{(\partial_x p)^2dx}
%%   + \gamma \int_0^L{(L-x)(\partial_x)^2dx} = \int_0^L{f(L-x)pdx}.  \ee
%%   Then we integrate in time
%% \beal
%% \int_0^L{(L-x)p^2(T)dx} &- \int_0^L{(L-x)p_0^2dx} - \frac{1}{2}\int_0^T{\int_0^L{p^2dx}dt} + \frac{3}{2}\int_0^T{\int_0^L{(\partial_x p)^2 dx}dt} \\
%% &+ \gamma \int_0^T{ \int_0^L{(L-x)(\partial_x p)^2dx}dt}  = \int_0^T{\int_0^L{f(L-x)p dx}dt}.
%% \eeal
%% This leads to the inequality
%% \beal
%% \frac{3}{2}\int_0^T{\int_0^L{(\partial_x p)^2dx}dt} \leq \int_0^T{\norm{f}_{L^2(\Omega)}\norm{(L-x)p}_{L^2(\Omega)}dt} + \frac{1}{2}\int_0^T{\norm{p}_{L^2(\Omega)}^2 dt} +  \int_0^L{(L-x)p_0^2dx}\\
%% \leq L \norm{p}_{C(0,T,L^2(\Omega))}\norm{f}_{L^1(0,T,L^2(\Omega))} + \frac{T}{2}\norm{p_0}_{L^2(\Omega)}^2 + L\norm{p_0}_{L^2(\Omega)}^2.
%% \eeal
%% Using \eqref{c0estimateadjoint} we write
%% \beal
%% \frac{3}{2}\int_0^T{\int_0^L{(\partial_x p)^2dx}dt} \leq L\left(\norm{p_0}_{L^2(\Omega)}+\norm{f}_{L^1(0,T,L^2(\Omega))} \right)\norm{f}_{L^1(0,T,L^2(\Omega))}+\left( \frac{T}{2}+L\right)\norm{p_0}_{L^2(\Omega)}^2
%% \eeal
%% A final use of Young's inequality allows to claim the existence of a constant $C(T,L)$ such that \eqref{linestimateadjoint} is satisfied.
%\end{proof}


%\subsubsection{Banach-fixed point theorem}
%We are now interested in the whole problem
%\bealn
%&\partial_t p - A^*p = f + y\partial_x p \mbox{ in } I\times\Omega\\
%&\tilde{p}(.,0) = \tilde{p}(.,L) = \partial_x \tilde{p} (.,0) = 0 \mbox{ on } I,\\
%&p(0,.) = p_0(.)
%\label{fulladjoint}
%\eealn
%As in the case of the state or the tangent equation, we proceed with a Banach fixed point argument. We define the linear mapping
%\beal
%& \Psi_{p_0,y}\colon \mathcal B_{\theta}\rightarrow \mathcal B_{\theta},\\
%& \Psi_{p_0,y}(p)= \mathcal L(p_0,f + y\partial_x p)
%\eeal
%where $\mathcal L$ shall be the solution operator of the nonhomogeneous linear system \eqref{nonhomoadjoint1} - \eqref{nonhomoadjoint3}, whose existence is guaranteed by Proposition~\ref{existencenonhomoadjoint}. Now we state a lemma that will enable us to prove that $\Psi_{p_0,y}$ is a contraction mapping on some well chosen ball $\mathcal{V}$, thus paving the way for a Banach fixed point theorem.
\begin{lem}\label{lemadjoint}
  Let $y\in\mathcal B$, $p\in\mathcal B$. Then it holds
\be
\norm{y\partial_x p}_{L^1(0,T,L^2(\Omega))} \leq c\,T^{1/4}\norm{y}_{\mathcal{B}}\norm{p}_{\mathcal{B}}.
\ee
\end{lem}
\begin{proof}[Proof of Lemma~\ref{lemadjoint}]
\beal
  \norm{y\partial_x p}_{L^1(0,T,L^2(\Omega))}&\leq \int_0^T{\norm{y}_{L^{\infty}(\Omega)}\norm{p}_{H^1_0(\Omega)}}~\mathrm dt  \leq c\,\int_0^T{\norm{y}_{L^2(\Omega)}^{1/2}\norm{y}_{H^1_0(\Omega)}^{1/2}\norm{p}_{H^1_0(\Omega)}}~\mathrm dt\\
&\leq c\,\norm{y}_{C(0,T,L^2(\Omega))}^{1/2}\left( \int_0^T{\norm{y}_{H^1_0(\Omega)}}~\mathrm dt\right)^{1/2}\norm{p}_{L^2(I,H^1_0(\Omega))}\\
& \leq c\, T^{1/4} \norm{y}_{C(0,T,L^2(\Omega))}^{1/2} \norm{y}_{L^2(0,T,H^1_0(\Omega))}^{1/2}\norm{p}_{L^2(0,T,H^1_0(\Omega))},
\eeal
which implies the assertion.
\end{proof}
% We can define the linear mapping
% \beal
% & \Psi_{\theta_1,\theta_2}\colon \mathcal B_{\theta_1,\theta_2}\rightarrow \mathcal B_{\theta_1,\theta_2},\\
% & \Psi_{\theta_1,\theta_2}(p)= \mathcal L_{nh}(-(y - y_d) + y\partial_x p)
% \eeal
% where $\mathcal L_{nh}$ is the solution operator of the nonhomogeneous linear system \eqref{nonhomoadjoint}.
% Thanks to Proposition~\ref{existencenonhomoadjoint} and Lemma~\ref{lemadjoint} we have
% \beal
% \norm{ \Psi_{\theta_1,\theta_2}(p)}_{\mathcal{B}_{\theta_1, \theta_2}} &\leq C \left( \norm{p_0}_{L^2(\Omega)} + \norm{y - y_d}_{L^1(0,T,L^2(\Omega))} + \norm{y \partial_x p}_{L^1(0,T,L^2(\Omega))} \right)\\
% &\leq C \left( \norm{p_0}_{L^2(\Omega)} + \norm{y - y_d}_{L^1(0,T,L^2(\Omega))} + T^{1/4}\norm{y}_{\mathcal{B}}\norm{p}_{\mathcal{B}} \right)
% \eeal
% and
% \beal
% \norm{ \Psi_{\theta_1,\theta_2}(p_1) - \Psi_{\theta_1,\theta_2}(p_2) }_{\mathcal{B}_{\theta_1, \theta_2}} & \leq C\norm{y\partial_x(p_1 - p_2)}_{L^1(0,T,L^2(\Omega))}\\
% & \leq C T^{1/4}\norm{y}_{\mathcal{B}}\norm{p_1 - p_2}_{\mathcal{B}}.
% \eeal
% Then we choose $r = 2C \left(\norm{p_0}_{L^2(\Omega)} + \norm{y - y_d}_{L^1(0,T,L^2(\Omega))}\right)$ and introduce the ball
% \be
% B_{r,\theta_1,\theta_2}=\{\delta y\in \mathcal B_{\theta_1,\theta_2}\colon \|p\|_{\mathcal B_{\theta_1,\theta_2}}\leq r\}.
% \ee
% Next we choose a $\tau=\theta_2-\theta_2>0$ small enough such that
% \[
% C\tau^{1/4}\|y\|_{\mathcal B}^2\leq\frac{1}{3}
% \]
% holds. Then the following estimates hold
% \[
% \|\Psi(p)_{\theta_1,\theta_2}\|_{\mathcal B_{\theta_1,\theta_2}}\leq \frac{5}{6}r,~~\|\Psi(\delta y_1)_{\theta_1,\theta_2}-\Psi(\delta y_2)_{\theta_1,\theta_1}\|_{\mathcal B_{\theta_1,\theta_2}}\leq\frac{2}{3}\|p_1-p_2\|_{\mathcal B_{\theta_1,\theta_2}}.
% \]
% which implies that $\Psi_{\theta_1,\theta_2}$ is a contraction mapping. So we can apply the Banach fixed point theorem blablabla.

\section*{Acknowledgement}
The authors warmly thank Konstantin Pieper and Boris Vexler for fruitful discussions during this work. Also, the authors gratefully acknowledge support from the International Research Training Group IGDK 1754, funded by the German Science Foundation (DFG).

%%% Local Variables:
%%% mode: latex
%%% TeX-master: "kdv"
%%% reftex-default-bibliography: ("~/Dropbox/KDV/Notes/kdvbib.bib")
%%% End: 