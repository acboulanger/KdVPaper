%!TEX root = kdv.tex
\appendix
\section{Well-posedness of the state equation, tangent equation and adjoint equation}
\label{sec:appwp}
\subsection{Linear estimates: Proof of estimate \eqref{linestimate_regular}}
\label{sec:linear-estimates}
  The proof is largely inspired from \cite{rosier1997exact,glass2008some}. Let $y\in \mathcal C(\bar I,\mathcal D(A))\,\cap\,\mathcal C^1(\bar I,L^2(\Omega))$ be the classical solution of \eqref{kdvlinnonhom} for smooth versions of the data $f$ and $y_0$. We multiply \eqref{kdvlinnonhom1} which holds in $L^2(\Omega)$ for a.e. $t\in I$ with $y$ and get
   \be
  \frac{1}{2}\frac{d}{dt}\int_{0}^{L}{y^{2}~\mathrm dx} + \abs{\partial_{x}y(t,0)}^{2} + \gamma \int_{0}^{L}{(\partial_{x} y)^{2}~\mathrm dx}=  \langle f,y\rangle_{H^{-1}(\Omega),H^{1}_{0}(\Omega)}.
  \ee
  Applying the Cauchy-Schwarz followed by Young's inequality to the right-hand side leads to
  \be \frac{1}{2}\frac{d}{dt}\int_{0}^{L}{y^{2}~\mathrm dx} + \abs{\partial_{x} y(t,0)}^{2} + \gamma \int_{0}^{L}{(\partial_{x}y)^{2}~\mathrm dx}\leq \frac{1}{2}\norm{f}_{H^{-1}(\Omega)}^{2} + \frac{1}{2}\norm{y}_{H^{1}_{0}(\Omega)}^{2}
  \label{1linnhupperbound}.
  \ee
  We proceed in the same manner testing with $xy$
  \be
  \frac{1}{2}\frac{d}{dt}\int_{0}^{L}{xy^{2}~\mathrm dx}-\frac{1}{2}\int_{0}^{L}{y^{2}~\mathrm dx} +  \frac{3}{2}\int_{0}^{L}{(\partial_{x} y)^{2}~\mathrm dx} +\gamma
  \int_{0}^{L}{x(\partial_{x} y)^{2}~\mathrm dx}= \langle f,xy \rangle_{H^{1}_{0}(\Omega)}.
  \label{2linnhupperbound}
  \ee
  The right-hand side is treated using the continuity of the duality pairing and Young's inequality
  {\color{red}\beal\label{qupperbound}
  \langle f,xy\rangle_{H^{-1}(\Omega),H^{1}_{0}(\Omega)} &\leq \norm{q}_{H^{-1}(\Omega)}\norm{\partial_x(xy)}_{L^2(\Omega)}\\
  & \leq \norm{f}_{H^{-1}(\Omega)}\norm{y + x\partial_{x}y}_{L^{2}(\Omega)}\\
  & \leq \norm{f}_{H^{-1}(\Omega)} \left( \norm{y}_{L^{2}(\Omega)} + L\norm{\partial_{x}y}_{L^{2}(\Omega)} \right)\\
  &\leq \frac{1}{2}\norm{f}_{H^{-1}(\Omega)}^{2} + \frac{1}{2}\norm{y}_{L^{2}(\Omega)}^{2} + \frac{L^{2}}{2}\norm{f}_{H^{-1}(\Omega)}^{2} + \frac{1}{2}\norm{\partial_{x}y}_{L^{2}(\Omega)}^{2}\\
  &\leq \frac{1+L^{2}}{2}\norm{f}_{H^{-1}(\Omega)}^{2}+\frac{1}{2}\norm{y}_{L^{2}(\Omega)}^{2}+\frac{1}{2}\norm{\partial_{x}y}_{L^{2}(\Omega)}^{2}.
  \eeal}
  Adding \eqref{1linnhupperbound}, \eqref{2linnhupperbound} (with the upper bound \eqref{qupperbound}) and omitting $\abs{\partial_{x} y(t,0)}^{2}$ on the left-hand side yields
  \[
  \frac{1}{2}\frac{d}{dt}\int_{0}^{L}{(1+x)y^{2}~\mathrm dx} +\left(\frac{1}{2}+\gamma\right)\|y\|_{H^1_0(\Omega)}^2\leq \left(1 +\frac{L^{2}}{2}\right)\norm{f}_{H^{-1}(\Omega)}^{2} + \|y\|_{L^2(\Omega)}^2.
  \]
  After integration between $0$ and $0<t<T$ we have
  \begin{multline*}
  \|y(t)\|_{L^2(\Omega)}^2+ \left( 1+2\gamma \right)\int_{0}^{t}{\|y\|_{H^1_0(\Omega)}^2~\mathrm dt}\leq \left(2+L^{2}\right)\int_{0}^{t}{\norm{f}_{H^{-1}(\Omega)}^{2}~\mathrm dt}\\
  +\|y_{0}\|_{L^2(\Omega)}^2+2\int_0^t{\|y\|_{L^2(\Omega)}^2~\mathrm dt}.
  \end{multline*}
  Then the standard Gronwall inequality gives us
  \[
  \|y(t)\|_{L^2(\Omega)}^2  \leq e^{2t}\left(\left(2+L^{2}\right)\norm{f}_{L^{2}(I,H^{-1}(\Omega))}^{2}+\norm{y_{0}}_{L^{2}(\Omega)}\right).
  \]
  This yields
  \[
  \|y\|_{\mathcal C(\bar I,L^2(\Omega))}+\|y\|_{L^2(I,H^1_0(\Omega))}\leq c(T,L)\,(\|f\|_{\Hm1}+\|y_0\|_{L^2(\Omega)})
  \]
  for some $c(T,L)>0$ independent of $y$, $f$ and $y_o$.
\subsection{Nonlinear estimates: Proof of Lemma~\ref{lemyyx2}}
\label{sec:nonl-state-equat}
The proof is inspired from \cite[Theorem 2.8]{faminskii2010initial}. Let us consider $y \in \B$ and $z \in \B$. There holds $\mathcal B\hookrightarrow L^4(I\times \Omega)$ and therefore we can estimate using $\|y\|_{L^\infty(\Omega)}^2\leq c\|y\|_{L^2(\Omega)}\|y\|_{H^1_0(\Omega)}$
\begin{multline*}
\norm{y\partial_x y - z\partial_x z}_{\Hm1}= \frac 1 2\left(\int_0^T{\left(\ \sup_{\norm{\varphi}_{H^1_0(\Omega)} = 1}(y^2 -  z^2,\partial_x\varphi)_{L^2(\Omega)}\right)^2}\mathrm dt\right)^{1/2} \\
\leq \frac{1}{2}\norm{z-y}_{L^4(I\times \Omega)}\norm{z+y}_{L^4(I\times \Omega)}\\
\leq\frac{1}{2} \norm{z-y}_{C(\bar I,L^2(\Omega))}^{1/2}\norm{z-y}_{L^2(I,L^{\infty}(\Omega))}^{1/2}\norm{z+y}_{C(\bar I,L^2(\Omega))}^{1/2}\norm{z+y}_{L^2(I,L^{\infty}(\Omega))}^{1/2}\\
\leq c\, \norm{z-y}_{C(\bar I,L^2(\Omega))}^{1/2}\norm{z+y}_{C(\bar I,L^2(\Omega))}^{1/2}\\
\left(\int_0^T{ \norm{z-y}_{L^2(\Omega)}\norm{z-y}_{H^1_0(\Omega)}~\mathrm dt}\right)^{1/4}\left(\int_0^T{ \norm{z+y}_{L^2(\Omega)}\norm{z+y}_{H^1_0(\Omega)}~\mathrm dt}\right)^{1/4}\\
\leq c \,T^{1/4}\norm{z-y}_{C(\bar I,L^2(\Omega))}^{3/4}\norm{z-y}_{L^2(I,H^1_0(\Omega))}^{1/4}\norm{z+y}_{C(\bar I,L^2(\Omega))}^{3/4}\norm{z+y}_{L^2(I,H^1_0(\Omega))}^{1/4} \\
\leq c \,T^{1/4}\norm{y-z}_{\B}\norm{y+z}_{\B}
\end{multline*}
\subsection{The tangent equation}\label{appendixtangent}
Next we analyze the well-posedness of the tangent equation.
\begin{subequations}
 \begin{numcases}{}
\partial_t \delta y +\partial_x \delta y + \partial_{xxx} \delta y - \gamma \delta \partial_{xx} y  + \partial_x(y \delta y)=  \delta u \mbox{ in } I\times\Omega,\label{linkdv1}\\
\delta y(\cdot,0) = \delta y(\cdot,L) = \partial_x \delta y (\cdot,L) = 0 \mbox{ in } I,\label{linkdv2}\\
\delta y(0,x) = \delta y_0 \mbox{ in } \Omega.\label{linkdv3}
 \end{numcases}
\end{subequations}
\begin{definition}
Let $(\delta u,\delta y_0)\in \Hm1\times L^2(\Omega)$ and $y\in \mathcal B$. A function $\delta y\in \mathcal B$ is called a solution of \eqref{linkdv1}-\eqref{linkdv3} if it satisfies the fixed point equation
\[
\delta y=\mathcal L(\delta u-\partial_x(y\delta y),\delta y_0)
\]
where $\mathcal L$ is the solution operator from Remark \ref{rmklinearoperator}.
\end{definition}
\begin{proposition}\label{prop:tangent}
 Let $(\delta u,\delta y_0) \in \Hm1\times L^2(\Omega)$ and $y\in \mathcal B$. Then, there exists a unique solution $\delta y \in \mathcal{B}$ of \eqref{linkdv1}-\eqref{linkdv3}. Furthermore, there exists a constant $\widetilde C(T,L,\norm{y}_{\mathcal{B}})$ such that the following estimate holds
 \be\label{estimatetangent}
 \norm{\delta y}_{\mathcal B}\leq \widetilde C \left( \norm{\delta y_0}_{L^2(\Omega)} + \norm{\delta u}_{\Hm1}\right).
 \ee
\end{proposition}
%[Proof of Proposition~\ref{prop:tangent}]
\begin{proof}
{\color{red}
We define the linear mapping
\[
\Psi_{\delta u, \delta y_0,y}\colon \mathcal B_{\theta}\rightarrow \mathcal B_{\theta},~~\Psi_{\delta u, \delta y_0,y}(\delta y) = \mathcal{L}(\delta u - \partial_x(y\delta y),\delta y_0)
\]
with $\mathcal B_{\theta}$ defined as in \eqref{btheta} and \eqref{normbtheta} and $\mathcal{L}$ being the linear \KdV operator described in Remark~\ref{rmklinearoperator}. Our goal is to show that under some constraints on $\theta$, $\Psi_{\delta u, \delta y_0,y}$ is a contraction mapping, such that the Banach fixed point theorem can be applied. First we estimate $\partial_x(y \delta y)$ in the $L^2((0,\theta),H^{-1}(\Omega))$-norm
\begin{multline}\label{estimate_variable_coefficient}
\|\partial_x(y\delta y)\|_{L^2((0,\theta),H^{-1}(\Omega))}=\left(\int_{0}^{\theta}\left(\underset{\|v\|_{H^1_0(\Omega)}=1}{\operatorname{sup}}(y\delta y,\partial_x v)_{L^2(\Omega)}\right)^2~\mathrm dt\right)^{1/2}\\
\leq\left(\int_{0}^{\theta}\|\delta y\|_{L^2(\Omega)}^2\|y\|_{L^\infty(\Omega)}^2~\mathrm dt\right)^{1/2}\leq c\,\|y\|_{\mathcal C([0,\theta],L^2(\Omega))}\left(\int_{0}^{\theta}\|\delta y\|_{H^1_0(\Omega)}\|\delta y\|_{L^2(\Omega)}~\mathrm dt\right)^{1/2}\\
\leq c\,\theta^{1/4}\,\|y\|_{\mathcal C([0,\theta],L^2(\Omega))}\|\delta y\|_{\mathcal C([0,\theta],L^2(\Omega))}^{1/2}\|\delta y\|_{L^2((0,\theta),H^1_0(\Omega))}^{1/2}\\
\leq c\,\theta^{1/4}\,\|y\|_{\mathcal B_{\theta}}\|\delta y\|_{\mathcal B_{\theta}}
\end{multline}
Therefore we can estimate
\begin{align*}
\|\Psi(\delta y)_{\delta u, \delta y_0,y}\|_{\mathcal B_{\theta}} & \leq \widetilde{C}\left(\|\delta y_0\|_{L^2(\Omega)}+\|\delta u\|_{L^2(I,H^{-1}(\Omega))}+\|\partial_x(y\delta y)\|_{L^2((0,\theta),H^{-1}(\Omega))}\right)\\
&\leq \widetilde{C}\left(\|\delta y_0\|_{L^2(\Omega)}+\|\delta u\|_{L^2(I,H^{-1}(\Omega))}\right) + \hat C\theta^{1/4}\|y\|_{\mathcal B}\|\delta y\|_{\mathcal B_{\theta}}
\end{align*}
and
\[
\|\Psi_{\delta u, \delta y_0,y}(\delta y_1)-\Psi_{\delta u, \delta y_0,y}(\delta y_2)\|_{\mathcal B_{\theta}}\leq \hat C\theta^{1/4}\|y\|_{\mathcal B}\|\delta y_1-\delta y_2\|_{\mathcal B_{\theta}}.
\]
Now we set $r=3 \widetilde{C}\left(\|\delta y_0\|_{L^2(\Omega)}+\|\delta u\|_{L^2(I,H^{-1}(\Omega))}\right)$ and introduce the ball
\[
B=\{\delta y \in \mathcal B_{\theta}\colon \|\delta y\|_{\mathcal B_{\theta}}\leq r\}.
\]
Next we choose $\theta$ small enough such that
\begin{equation}\label{linstatetheta}
\hat C\theta^{1/4}\|y\|_{\mathcal B}=\frac{1}{3}
\end{equation}
holds. Then the following inequalities hold
\[
\|\Psi_{\delta u, \delta y_0,y}(\delta y)\|_{\mathcal B_{\theta}}\leq \frac{2}{3}r,~~\|\Psi_{\delta u, \delta y_0,y}(\delta y_1) - \Psi_{\delta u, \delta y_0,y}(\delta y_2)\|_{\mathcal B_{\theta}}\leq\frac{1}{3}\|\delta y_1-\delta y_2\|_{\mathcal B_{\theta}}
\]
which imply that $\Psi_{\delta u, \delta y_0,y}$ is a contraction mapping on $B$. So we can apply the Banach fixed point theorem which guarantees the existence of a unique fixed point $\delta y$ of $\Psi_{\delta u, \delta y_0,y}$ which is a solution of \eqref{linkdv1}-\eqref{linkdv3} in $(0,\theta)$ with initial value $\delta y_0$. This strategy is repeated successively using the intermediate values of $y$ as initial data. The final time $T$ can be reached since the length $\theta$ of the existence intervals is independent of initial data, see \eqref{linstatetheta}. The concatenation of all $\delta y$ for is a solution of \eqref{linkdv1} - \eqref{linkdv3}. This strategy generates a time grid $(t_k)_{k=0}^N$ with $t_0=$ and $t_N=T$. Next we prove the estimate \eqref{estimatetangent}. Using the time-grid we get
\begin{multline*}
\|\delta y\|_{\mathcal B}\leq c\sum_{k=0}^{N-1}\left(\|\delta y\|_{\mathcal C([t_k,t_{k+1}],L^2(\Omega))}+\|\delta y\|_{L^2((t_k,t_{k+1}),H^1_0(\Omega))}\right)\\
\leq c\,N\left(\|\delta u\|_{\Hm1}+\|\delta y_0\|_{L^2(\Omega)}\right),
\end{multline*}
Then \eqref{linstatetheta} implies that there exists a $c>0$ such that
\[
N-1\leq\frac T\theta=cT\|y\|_{\mathcal B}^4
\]
and accordingly
\[
\|\delta y\|_{\mathcal B}\leq c(\|y\|_{\mathcal B})\,\left(\|\delta u\|_{\Hm1}+\|\delta y_0\|_{L^2(\Omega)}\right).
\]
Finally we discuss uniqueness. Let $\delta y\in \mathcal B$ be a solution of \eqref{linkdv1}--\eqref{linkdv3} for $\delta u=\delta y_0=0$. From the proof of \eqref{linestimate_regular} we see
\[
\|\delta y(t)\|_{L^2(\Omega)}^2\leq c\int_0^t\|\partial_x(y\delta y)(s)\|_{H^{-1}(\Omega)}^2~\mathrm ds\leq \tilde c\int_0^t\|y(s)\|_{H^1_0(\Omega)}^2\|\delta y(s)\|_{L^2(\Omega)}^2~\mathrm ds
\]
for any $T\geq t\geq 0$. Then Gronwall's Lemma implies $y\equiv0$. Thus existence of a unique solution is proven.}

%Concerning the estimates \eqref{estimatetangent}, the proof is very similar to the non-variable coefficients case (see Appendix~\ref{sec:linear-estimates}). We just mention the main differences.  In the case of  a smooth solution $\delta y \in \mathcal C(\bar I,\mathcal D(A))\cap \,\mathcal C^1(I,L^2(\Omega))$ we multiply \eqref{linkdv1} by $\delta y$ and estimate the term involving $y$ in the following way
%\begin{align*}
%\megaabs{\int_0^L{\delta y \partial_x \left( y \delta y \right)}~\mathrm dx} = \megaabs{-\int_0^L{y \delta y \partial_x \delta y}~\mathrm dx} &\leq \frac{1}{2\varepsilon}\int_0^L{y^2 \delta y^2}~\mathrm dx + \frac{\varepsilon}{2}\|\delta y\|_{H^1_0(\Omega)}^2\\
%&\leq \frac{1}{2\varepsilon}\norm{y}_{L^{\infty}(\Omega)}^2 \|\delta y\|^2_{L^2(\Omega)} + \frac{\varepsilon}{2}\|\delta y\|_{H^1_0(\Omega)}^2
%\end{align*}
%for any $\varepsilon>0$.
%In the same manner, multiplying \eqref{linkdv1} by $x\delta y$ leads to
%\begin{align*}
%\megaabs{\int_0^L{x\delta y \partial_x(y \delta y)}}~\mathrm dx &= \megaabs{- \int_0^Ly \delta y^2~\mathrm dx - \int_0^Lxy\delta y
%\partial_x \delta y~\mathrm dx}\\
%&\leq \left(\norm{y}_{L^{\infty}(\Omega)}+\frac{L^2}{2\varepsilon}\norm{y}_{L^{\infty}(\Omega)}^2\right)\|\delta y\|_{L^2(\Omega)}^2 +\frac{\varepsilon}{2}\|\delta y\|_{H^1_0(\Omega)}^2 \\
%\end{align*}
%for any $\varepsilon>0$. Based on these estimates the a priori estimate \eqref{estimatetangent} can be shown. The estimate for $\|\partial_t y\|_{L^2(I,\mathcal V^*)}$ can be shown based on \eqref{estimate_variable_coefficient} and \eqref{estimatetangent}.
\qquad\end{proof}

\subsection{The adjoint equation}
\label{appendixadjoint}
Next we study the following equation
\besn
-\partial_t p -\partial_x p - \partial_{xxx} p - \gamma \partial_{xx} p  - y\partial_x p =  \phi \mbox{ in } I\times\Omega,\label{adjoint1}\\
p(\cdot,0) = p(\cdot,L) = \partial_x p (\cdot,0) = 0 \mbox{ on } I,\label{adjoint2}\\
p(T) = p_{T} \mbox{ in } \Omega.\label{adjoint3}
\eesn
for any $y\in \mathcal B$.
\begin{definition}
A function $p\in \mathcal B$ is called a solution of \eqref{adjoint1}-\eqref{adjoint3} if it solves the fixed point equation
\[
p(t)=W^*(t)p_T+\int_t^TW^*(s-t)(\phi(s)+y(s)\partial_x p(s))~\mathrm ds.
\]
\end{definition}
\begin{lemma}\label{lemadjoint}
  Let $y\in\mathcal B$, $p\in\mathcal B$. Then it holds
\[
\norm{y\partial_x p}_{L^1(0,T,L^2(\Omega))} \leq c\,T^{1/4}\norm{y}_{\mathcal{B}}\norm{p}_{\mathcal{B}}.
\]
\end{lemma}
\begin{proof}
We estimate
\begin{multline*}
\norm{y\partial_x p}_{L^1(I,L^2(\Omega))}\leq \int_0^T{\norm{y}_{L^{\infty}(\Omega)}\norm{p}_{H^1_0(\Omega)}}~\mathrm dt\\
\leq c\,\int_0^T{\norm{y}_{L^2(\Omega)}^{1/2}\norm{y}_{H^1_0(\Omega)}^{1/2}\norm{p}_{H^1_0(\Omega)}}~\mathrm dt\\
\leq c\,\norm{y}_{C(\bar I,L^2(\Omega))}^{1/2}\left( \int_0^T{\norm{y}_{H^1_0(\Omega)}}~\mathrm dt\right)^{1/2}\norm{p}_{L^2(I,H^1_0(\Omega))}\\
\leq c\, T^{1/4} \norm{y}_{C(\bar I,L^2(\Omega))}^{1/2} \norm{y}_{L^2(I,H^1_0(\Omega))}^{1/2}\norm{p}_{L^2(I,H^1_0(\Omega))},
\end{multline*}
which implies the assertion.
\qquad\end{proof}
\begin{proposition}
Let $(\phi,p_T)\in L^1(I,L^2(\Omega))\times L^2(\Omega)$. Then the equation \eqref{adjoint1}-\eqref{adjoint3} has a unique solution $p\in \mathcal B$. Furthermore there exists a constant $c\,(\|y\|_{\mathcal B})>0$ such that
\begin{equation}\label{apriori_adjoint}
\|p\|_{\mathcal B}\leq c\,(\|p_T\|_{L^2(\Omega)}+\|\phi\|_{L^1(I,L^2(\Omega))})
\end{equation}
holds.
\end{proposition}
\begin{proof}
{\color{red}Uniqueness follows from \cite[Proposition 15]{coron2003exact}. Existence and the estimate \eqref{apriori_adjoint} are proven in the same way as in the proof of Proposition \ref{prop:tangent}. Hereby we rely on Lemma \ref{lemadjoint}.}
%The proof uses similar arguments as the proof of Proposition \ref{prop:tangent}. In particular it is based on the Banach fixed theorem and Lemma \eqref{lemadjoint}. The estimate \eqref{apriori_adjoint} follows from
%\begin{multline*}
%\left|\int_0^Ly\partial_xpp~\mathrm dx\right|=\left|\frac 1 2\int_0^Lp^2\partial_xy~\mathrm dx\right|\leq c\,\|p\|_{H^1_0(\Omega)}\|p\|_{L^2(\Omega)}\|y\|_{H^1_0(\Omega)}\\
%\leq \frac{c}{2\varepsilon}\|p\|_{L^2(\Omega)}^2\|y\|_{H^1_0(\Omega)}^2+\frac{\varepsilon}{2}\|p\|_{H^1_0(\Omega)}^2
%\end{multline*}
%for any $\varepsilon>0$ and
%\begin{multline*}
%\left|\int_0^Ly\partial_xp(L-x)p~\mathrm dx\right|=\left|\frac 1 2\int_0^L(L-x)p^2\partial_xy-yp^2~\mathrm dx\right|\\
%\leq c\,\left(\|p\|_{H^1_0(\Omega)}\|p\|_{L^2(\Omega)}\|y\|_{H^1_0(\Omega)}+\|p\|_{L^2(\Omega)}^2\|y\|_{H^1_0(\Omega)}\right)\\
%\leq \|p\|_{L^2(\Omega)}^2\left(\frac{c}{2\varepsilon}\|y\|_{H^1_0(\Omega)}^2+\|y\|_{H^1_0(\Omega)}\right)+\frac{\varepsilon}{2}\|p\|_{H^1_0(\Omega)}^2.
%\end{multline*}
%for any $\varepsilon>0$.
\qquad\end{proof}





%\paragraph{\textbf{Acknowledgement}}
%\section{Acknowledgement}
%The authors warmly thank Konstantin Pieper and Boris Vexler for fruitful discussions during this work. Also, the authors gratefully acknowledge support from the International Research Training Group IGDK 1754, funded by the German Science Foundation (DFG).


%%% Local Variables:
%%% mode: latex
%%% TeX-master: "kdv"
%%% reftex-default-bibliography: ("~/Dropbox/KDV/Notes/kdvbib.bib")
%%% End: 