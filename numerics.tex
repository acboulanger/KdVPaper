%!TEX root = kdv.tex
%%%%%%%%%%%%%%%%%%%%%%%
\section{Discretization and numerical results}
%%%%%%%%%%%%%%%%%%%%%%%
\label{secnum}
\subsection{Discretization of the Korteweg-de Vries-Burgers equation}
In this section we consider the discretization of the \KdVB equation on the domain $\Omega = [0,L]$. The difficulty here comes from the temporal scales. While the effects of the nonlinearity are only visible after a long time interval, the linear part involves a wide range of scales: the third order derivative represents in particular a high frequency term. We wish to explain in a few lines hereafter what reasons lead us to our final choice. In the case of bounded domains numerous schemes are available, be it finite differences \cite{djidjeli1995numerical,zabusky1965interaction}, finite elements \cite{winther1980conservative,arnold1982superconvergent}, finite volumes \cite{dutykh2013finite}, discontinuous Galerkin schemes \cite{Bona1986859,yan2002local}, or polynomial -~~collocation or Galerkin~~- spectral methods \cite{ma2000legendre,ma2001optimal,shen2003new}. With the aim of tackling an optimal control problem, spectral discretizations present interesting advantages compared to any finite difference or finite element method \cite{boydchebyshev}. First of all they provide very low approximation errors: in many cases, these approaches are exponentially convergent. As a result, the number of grid points required to achieve the desired precision can be very low and the memory needed to store the variables less than for alternative methods. This is a very convenient feature in our perspective, where the optimization part requires the storage of the forward and backward problem in space and time. Finally, there exist high-performance implementations of the algorithms required to transform bases for most spectral methods. Because of the nonlinearity in the \KdVB equation, spectral collocation methods shall be favoured: they indeed allow to represent the variables in terms of their values on a set of points and not only as the coefficients in the spectral expansion. However, for third order equation, the authors show in \cite{merryfield1993properties} that Chebyshev and Legendre collocation methods are unstable. That is why pseudospectral method habe been popular for third order PDEs. Fourier pseudospectral methods have been extensively used \cite{trefethen2000spectral,maday1988error,fornberg1978numerical}. They are in particular very fast when combined with the Fast Fourier Transform to treat the nonlinear term on collocation points. Nevertheless, they are only suited for periodic boundary conditions. Polynomial pseudospectral method on the contrary, which are more adequate for bounded intervals, proved to be stable for the linear third-order differential equation with Chebyshev or Legendre Gauss Lobatto points as collocation points \cite{HuangSloan1992}. They have also demonstrated great performances compared to any other method (see the benchmarking in \cite{skogestad2009boundary}). An interesting hybrid method is proposed in \cite{ma2000legendre}, where the linear term is treated by a Petrov-Galerkin method (better suited for nonsymmetric problems) on Legendre polynomials, while the nonlinear term is treated pseudospectrally using a Chebyshev method. Shortly after, Shen \cite{shen2003new} proposed another dual Petrov-Galerkin method with nearly optimal computational complexity. Moreover, it is also equivalent to a natural weighted spectral-Galerkin formulation. For those reasons, this will be our method of choice and we recall it here briefly. Since it has never been used on optimal control problem, we also discuss an appropriate discretization of the method regarding the time stepping scheme used.

\subsection{The dual Petrov-Galerkin method}
The main difference with the method in \cite{ma2000legendre} lies in the choice of the test and trial functions bases. They are chosen as a compact combination of Legendre polynomials in such a way that the trial functions satisfy the underlying boundary conditions of the equation and the test functions satisfy the dual boundary conditions. Therefore, most matrices involved in the resolution of the problem are sparse or well-conditioned. We present the method for some reference domain $[-1,1]$, but it can be extended to any other domain of type $[a,b]$ by carefully scaling the Legendre polynomials and the integrals. We denote by $P_N$ the space of polynomials of degree $\leq N$ and set
\be
V_N = \left\{y\in P_N : y(1)=y(-1)=\partial_x y(1)=0\right\},
\ee
\be
V_N^{\ast} = \left\{y\in P_N : y(1) = y(-1) = \partial_x y(-1)=0\right\}.
\ee
We consider then the semi-discrete problem:
Find $y_N \in V_N$
\begin{multline}
\langle\partial_t y_N, \varphi_N\rangle - \langle y_N, \partial_x \varphi_N \rangle + \langle\partial_x y_N, \partial_{xx}\varphi_N \rangle  + \\ \gamma \langle \partial_x y_N, \partial_x \varphi_N \rangle- \langle\frac{y_N^2}{2}, \partial_x \varphi_N\rangle= \langle q, \varphi_N\rangle, \quad \forall v_N \in V_N^{\ast}
\label{petrovgalerkin}
\end{multline}

\subsubsection{Basis functions}
Let us denote by $L_k$ the $k$th Legendre polynomial, and define the basis functions as follows (see Figure~\ref{basisfunctions})
\be
\phi_k(x) = L_k(x) - \frac{2k+3}{2k+5}L_{k+1}(x) - L_{k+2}(x) + \frac{2k+3}{2k+5}L_{k+3}(x),
\ee
\be
\psi_k(x) = L_k(x) + \frac{2k+3}{2k+5}L_{k+1}(x) - L_{k+2}(x) - \frac{2k+3}{2k+5}L_{k+3}(x).
\ee
\begin{figure}[h!]
\begin{center}
\begin{tikzpicture}[scale=0.7]
\begin{axis}[title={\Large First test functions},
	xmin=-50,
        xmax = 50,
        ymin = -1.5,
        ymax = 2,
        xlabel=$x$,
        legend entries={
        $\psi_0$,
	$\psi_1$,
	$\psi_2$,
	$\psi_5$,
	$\psi_{10}$,
    },
    legend pos=outer north east,
    legend cell align=left,]
\pgfplotstableread{testbasis.txt}\mydata;
\addplot [ 
           color=red,thick,
         ]
         table
         [
           x expr=\thisrowno{0}, 
           y expr=\thisrowno{1} 
         ] {\mydata};
 \addplot [ 
           color=blue,thick,
         ]
         table
         [
           x expr=\thisrowno{0}, 
           y expr=\thisrowno{2} 
         ] {\mydata};
\addplot [ 
           color=green,thick,
         ]
         table
         [
           x expr=\thisrowno{0}, 
           y expr=\thisrowno{3} 
         ] {\mydata};
\addplot [ 
           color=orange,thick,
         ]
         table
         [
           x expr=\thisrowno{0}, 
           y expr=\thisrowno{4} 
         ] {\mydata};
\addplot [ 
           color=cyan,thick,
         ]
         table
         [
           x expr=\thisrowno{0}, 
           y expr=\thisrowno{5} 
         ] {\mydata};
\end{axis}
\end{tikzpicture}
\hspace{0.5cm}
\begin{tikzpicture}[scale=0.7]
\begin{axis}[title={\Large First trial functions},
	xmin=-50,
        xmax = 50,
        ymin = -1.5,
        ymax = 2,
        xlabel=$x$,
        legend entries={
        $\phi_0$,
	$\phi_1$,
	$\phi_2$,
	$\phi_5$,
	$\phi_{10}$,},
    legend pos=outer north east,
    legend cell align=left,]
\pgfplotstableread{trialbasis.txt}\mydata;
\addplot [ 
           color=red,thick,
         ]
         table
         [
           x expr=\thisrowno{0}, 
           y expr=\thisrowno{1} 
         ] {\mydata};
 \addplot [ 
           color=blue,thick,
         ]
         table
         [
           x expr=\thisrowno{0}, 
           y expr=\thisrowno{2} 
         ] {\mydata};
\addplot [ 
           color=green,thick,
         ]
         table
         [
           x expr=\thisrowno{0}, 
           y expr=\thisrowno{3} 
         ] {\mydata};
\addplot [ 
           color=orange,thick,
         ]
         table
         [
           x expr=\thisrowno{0}, 
           y expr=\thisrowno{4} 
         ] {\mydata};
\addplot [ 
           color=cyan,thick,
         ]
         table
         [
           x expr=\thisrowno{0}, 
           y expr=\thisrowno{5} 
         ] {\mydata};
\end{axis}
\end{tikzpicture}

  \caption{Basis functions for the Petrov-Galerkin method.}
\label{basisfunctions}
 \end{center}
\end{figure}


Thus for $N \geq 3$,
\beal
&V_N = \span \left\{ \phi_0,\phi_1,...,\phi_{N-3}\right\},\\
&V_N^{\ast} = \span \left\{ \psi_0,\psi_1,...,\psi_{N-3}\right\}.
\eeal
Then, setting
\beal
& y_N = \sum_{k=0}^{N-3}{\tilde y_k\phi_k},\quad\bar y = \left( \tilde y_0, \tilde y_1, ..., \tilde y_{N-3}\right)^t\\
& q_N = \sum_{k=0}^{N-3}{\tilde q_k\phi_k},\quad \bar q = \left( \tilde q_0, \tilde q_1, ..., \tilde q_{N-3}\right)^t\\
& m_{ij}=(\phi_j, \psi_i),\quad p_{ij}=-(\phi_j^{'}, \psi_i),\quad q_{ij}=(\phi_j^{'}, \psi_i^{'}),\quad s_{ij}=(\phi_j^{'},\psi_i^{''}),
\eeal
the variational formulation \eqref{petrovgalerkin} yields
\be
M\partial_t \bar y + \left( -P +\gamma Q  + S \right)\bar y + F(\bar y) = M \bar q,
\ee
where $M$, $P$, $Q$ and $S$ are matrices of size $(N-2)\times(N-2)$ with coefficients $m_{ij}, p_{ij}, q_{ij}$ and $s_{ij}$. $F(\bar y)$ represents the nonlinear term and it is treated as suggested in \cite{shen2003new} i.e. using the pseudospectral approach. It means that the nonlinearity is evaluated in the spatial domain, that we choose to be the Chebyshev-Gauss-Lobatto(CGL) points, and transfered back in the Legendre spectral space. We therefore need to be able to transform back and forth from the spectral space of Legendre coefficients to the values on the CGL points. This can be done using the fast Fourier transform(FFT) and the Chebyshev-Legendre transform. However, for the polynomial degrees we consider here (between 160 and 512), we consider a direct method that is faster and easier to handle, especially when it will come to finding a discrete adjoint. We build beforehand the matrices  $L_1 =\left(\phi_j(x_i)\right)$ and $L_2 =\left(\psi_j(x_i)\right)$, $i=1..N+1$, $j=1..N-2$, where the points $x_i$ are the CGL points such that $L_1 \bar y = (y_N(x_1), y_N(x_2), ...y_N(x_{N+1}))^t$ and the same holds with $L_2$ for a variable in the dual space.

\subsection{Time stepping scheme and adjoint - Crank-Nicolson}
We have to deal with a problem of high order derivative. Therefore an explicit temporal discretization would lead to excessively small time steps in order to get stability. An implicit method should rather be considered. In \cite{li2000error}, the authors prove convergence of a pseudospectral method with backward Euler scheme for the \KdV equation. However in practice, the first order accuracy in time authorizes only very small time steps. A second order implicit scheme like the Crank-Nicolson scheme should be preferable, though the resolution of the nonlinear system is computationally demanding. This scheme also has the advantage of being a method of choice in optimal control: using the representation of the Crank-Nicolson scheme as a continuous Galerkin method of degree one (continuous trial linear functions and discontinuous piecewise constant test functions) allows us to give directly the concrete form of the adjoint, tangent and additional adjoint equations leading to the exact computation of the discrete gradient and Hessian \cite{meidner2007adaptive}. Note that the use of discrete derivatives is important for the convergence of our optimization algorithm. The scheme for the state is

\bealn
&M \tilde y_0=M y_0\\
&M \tilde y_{n+1} + \frac{\Delta t}{2}\left( (S +\gamma Q -P)\tilde y_{n+1} - F(\tilde y_{n+1})\right)  &=  M \tilde y_{n} + \frac{\Delta t}{2}\left( (S +\gamma Q-P)\tilde y_{n} - F(\tilde y_{n})\right)  \\
 &  &+ \frac{\Delta t}{2}\left( M\tilde q_n + M \tilde q_{n+1}\right), \quad n=0..N.
\eealn

Then the discrete adjoint scheme in case of distributed control writes
%\begin{equation}\left\{
%\begin{split}
%M^t p_{N+1} + \frac{\Delta t}{2}\left( (S^t +\gamma Q^t- P^t)p_{N+1} - F'(y_{N+1})^t p_{N+1}\right) = -\frac{\Delta t}{2} A(y_{N+1} - y_d)\\
% M^t p_{n-1} +  \frac{\Delta t}{2}\left( (S^t  +\gamma Q^t- P^t)p_{n-1}  - F'(y_{n-1})^t p_{n-1}\right)  &= M^t p_{n}\\
%&- \frac{\Delta t}{2}A((y_{n} - y_d)-(y_{n-1} - y_d))+\frac{\Delta t}{2}\left( (S^t +\gamma Q^t- P^t)p_{n} - F'(y_{n-1})^t p_{n}\right) ,\\
%&n=2..N+1
%\end{split}
%\right.
%\end{equation}
\bealn
&M^t p_{N+1} + \frac{\Delta t}{2}\left( (S^t +\gamma Q^t- P^t)p_{N+1} - F^{'}(y_{N+1})^t p_{N+1}\right) = -\frac{\Delta t}{2} A(y_{N+1} - y_d) \\
&M^t p_{n-1} +  \frac{\Delta t}{2}\left( (S^t  +\gamma Q^t- P^t)p_{n-1}  - F(y_{n-1})^t p_{n-1}\right)  = M^t p_{n} \\
& \mbox{\hspace{0.15\textwidth}}- \frac{\Delta t}{2}A((y_{n} - y_d)-(y_{n-1} - y_d) + \frac{\Delta t}{2}\left( (S^t +\gamma Q^t- P^t)p_{n} - F^{'}(y_{n-1})^t p_{n}\right) ,\\
& \mbox{\hspace{0.6\textwidth}}n=2..N+1\\
&M^t p_{0}  = M^t p_{1} + \frac{\Delta t}{2}\left( (S^t +\gamma Q^t- P^t)p_{1} - F^{'}(y_{0})^t p_{1}\right)-\frac{\Delta t}{2} A(y_{1} - y_d)
\eealn
where the matrix $A = \left( \langle \phi_i,\phi_j\rangle\right)$ comes from the discretization of the Lagrangian (in particular the $L^2$ norm in the cost function). We also give the case of the terminal observations problem because we will use it in our numerical examples.
\bealn
& M^t p_{N+1} + \frac{\Delta t}{2}\left( (S^t +\gamma Q^t- P^t)p_{N+1} - F^{'}(y_{N+1})^t p_{N+1}\right) = - A(y_{N+1} - y_d)\\
& M^t p_{n-1} + \frac{\Delta t}{2}\left( (S^t +\gamma Q^t- P^t)p_{n-1} - F^{'}(y_{n-1})^t p_{n-1}\right)  = M^t p_{n} + \\
& \mbox{\hspace{0.35\textwidth}}\frac{\Delta t}{2}\left( (S^t +\gamma Q^t- P^t)p_{n} - F^{'}(y_{n-1})^t p_{n}\right), \, n=2..N+1\\
&M^t p_{0}  = M^t p_{1} + \frac{\Delta t}{2}\left( (S^t +\gamma Q^t- P^t)p_{1} - F^{'}(y_{0})^t p_{1}\right)
 \eealn
 
 \subsection{Time stepping scheme and adjoint - Crank-Nicolson-Leap-Frog}
 An alternative to the Crank Nicolson scheme is the two-steps method Crank-Nicolson Leap Frog. In this setting, the third derivative is treated implicitely and the nonlinear term is treated explicitely. This method has already been extensively used for the \KdV equation \cite{shen2003new,ma2000legendre,ma2001optimal}, showing for instance extended stability intervals with the Fourier spectral method \cite{chan1985fourier}. However, to the authors knowledge, this has never been used in the context of an optimal control problem. Commonly, this method is initialized by a semi-implicit step. We suggest also a slight modification of the last step of this two-steps method in order to get a discrete adjoint that is consistent with the continuous adjoint in both the distributed control problem or the terminal observations problem. Numerical experiments are also available in the next section. The forward scheme is as follows
\bealn
& \frac{1}{2}\left( M+\Delta t S\right) y_1 = \frac{1}{2}\left( M + \Delta t P + \gamma \Delta t Q \right) y_0 + \frac{1}{2}\Delta t F(y_0) + \frac{1}{2}\Delta t M q_0 \\
& \frac{1}{2}\left( M+\Delta t S\right) y_{n+1} = \frac{1}{2}\left( M - \Delta t S\right) y_{n-1} +  \Delta t \left( P + \gamma Q\right)y_n \\
& \mbox{\hspace{0.5\textwidth}}+ \Delta t F(y_n) + \Delta t M q_n,  \, n=1..N-1\\
& M y_{N+1} = \frac{1}{2}\left( M y_N + M y_{N-1} \right)+ \frac{1}{2}\Delta t P y_N + \frac{1}{2}\Delta t F(y_N) - \frac{1}{2}\Delta t S y_{N-1}
\eealn
The last step is not a regular well-known step but it is designed in such a way that the discrete adjoint shall be consistent with a CNLF discretization of the continuous adjoint. Notice that as $\Delta t$ vanishes, it becomes just a mean between the two preceding steps. The discrete adjoint in the case of terminal observations then writes
 \bealn
 &M^t p_N = -A(y_{N+1} - y_d)\\
 &\frac{1}{2}\left( M^t+\Delta t S^t\right) p_{N-1} = \frac{1}{2}\left( M^t + \Delta t P^t + \gamma \Delta t Q^t \right) p_N + \frac{1}{2}\Delta t F^{'}(y_N)^t p_N \\
 &\frac{1}{2}\left( M^t + \Delta t S^t\right)p_{n-2} = \frac{1}{2}\left( M^t - \Delta t S^t\right)p_{n} + \Delta t \left( P^t + \gamma Q^t\right)p_{n-1} + \Delta t F^{'}(y_{n-1})^t p_{n-1}, \\
 &\mbox{\hspace{0.7\textwidth}}\quad n=2..N.
\eealn
\begin{proof}
The lagrangian of the problem writes at the continuous level.
\beal
  \mathcal{L}(y,q,p) = \frac{1}{2}\norm{y - y_d}_{\lspace}^2 + \alpha \norm{q}_{\M} + <p, By-q>
\eeal
Its (almost) discrete counterpart shall be
\begin{multline}
L(\bar y, \bar q, \bar p) = \frac{1}{2}(y_{N+1} - y_d)^t A (y_{N+1} - y_d) + \alpha \norm{q}_{\M} \\
+ p_0^t\left[ \frac{1}{2}\left( M+\Delta t S\right) y_1 - \frac{1}{2}\left( M + \Delta t P + \gamma \Delta t Q \right) y_0 - \frac{1}{2}\Delta t F(y_0) - \frac{1}{2}\Delta t M q_0 \right]\\
+\sum_{k=1}^{N-1}{p_k^t\left[
\frac{1}{2}\left( M+\Delta t S\right) y_{k+1} - \frac{1}{2}\left( M - \Delta t S\right) y_{k-1} - \Delta t \left( P + \gamma Q\right)y_k - \Delta t F(y_k) - \Delta t M q_k\right]}\\
+p_N^t\left[ M y_{N+1} - \frac{1}{2}\left( M y_N + M y_{N-1} \right)- \frac{1}{2}\Delta t P y_N - \frac{1}{2}\Delta t F(y_N) +\frac{1}{2}\Delta t S y_{N-1}\right].
\end{multline}
Differentiating with respect to $\bar y$ leads
\begin{multline}
\delta L(\bar y, \bar q, \bar p)\delta \bar y = \delta y_{N+1}^t A (\delta y_{N+1} - y_d) \\
+ p_0^t\left[ \frac{1}{2}\left( M+\Delta t S\right) \delta y_1 - \frac{1}{2}\left( M + \Delta t P + \gamma \Delta t Q \right) \delta y_0 - \frac{1}{2}\Delta t F'(y_0)\delta y_0\right]\\
+\sum_{k=1}^{N-1}{p_k^t\left[
\frac{1}{2}\left( M+\Delta t S\right) \delta y_{k+1} - \frac{1}{2}\left( M - \Delta t S\right) \delta y_{k-1} - \Delta t \left( P + \gamma Q\right)y_k - \Delta t F'(y_k)\delta y_k \right]}\\
+p_N^t\left[ M \delta y_{N+1} - \frac{1}{2}\left( M \delta y_N + M \delta y_{N-1} \right)- \frac{1}{2}\Delta t P \delta y_N - \frac{1}{2}\Delta t F'(y_N)\delta y_N + \frac{1}{2}\Delta tS\delta y_{N-1}\right]
\end{multline}
Symmetry of the scalar product yields, after re-indexing of the terms
\beal
\delta L(\bar y, \bar q, \bar p)\delta \bar y &= \delta y_{N+1}^t A (\delta y_{N+1} - y_d) \\
&+ \left[ \delta y_1^t\frac{1}{2}\left( M+\Delta t S\right)  - \delta y_0^t\frac{1}{2} \left( M + \Delta t P + \gamma \Delta t Q \right) - \delta y_0^t\frac{1}{2}\Delta t F^{'}(y_0)\right]p_0\\
&+\sum_{k=2}^{N}{\delta y_k^t\left[
\frac{1}{2}\left( M+\Delta t S\right)^t \right]p_{k-1}}-\sum_{k=0}^{N-2}{\delta y_k^t\left[\frac{1}{2}\left( M+\Delta t S\right)^t \right]p_{k+1}}\\
&-\sum_{k=1}^{N-1}{\left[ \delta y_k^t \Delta t\left(P^t + \gamma Q^t + F^{'}(y_k)^t\right)\right]p_k}\\
&+\delta y_{N+1}^t M^t p_N - \delta y_{N-1}^t \frac{1}{2}(M^t - \Delta t S^t)p_N \\
&-\delta y_{N}^t \left( \frac{1}{2}(M^t + \Delta t P^t + \Delta t F^{'}(y_N)^t) \right)p_N.
\eeal
The discrete adjoint scheme is obtained by imposing $\delta L(\bar y, \bar q, \bar p)\delta \bar y = 0$.
\end{proof}
 
 
 \begin{rmk}
 Without this modification of the last forward step, the discrete adjoint obtained in the case of terminal observation would be
 \bealn
 &\frac{1}{2}(M^t+\Delta t S^t) p_N = -A(y_{N+1} - y_d)\\
 &\frac{1}{2}\left( M^t+\Delta t S^t\right) p_{N-1} =  \left(\Delta t P^t + \gamma \Delta t Q^t \right) p_N + \Delta t F^{'}(y_N)^t p_N \\
 &\frac{1}{2}\left( M^t + \Delta t S^t\right)p_{n-2} = \frac{1}{2}\left( M^t - \Delta t S^t\right)p_{n} + \Delta t \left( P^t + \gamma Q^t\right)p_{n-1} \\
 &\mbox{\hspace{0.4\textwidth}}+ \Delta t F^{'}(y_{n-1})^t p_{n-1}, \quad n=2..N.
 \eealn
 The major difference lies in the second step. It does not correspond to a consistent discretization of the continuous adjoint equation: the time derivative is not reconstructed. It results in practice in huge oscillations in time of the adjoint variable.
 \end{rmk}

\subsection{Numerical treatment of the optimization problem}
Numerically, the optimization problem is solved via a Newton type method. This requires an additional regularization term which we introduce as follows
\be
\min_{q \in \M} J(y) = \frac{1}{2}\norm{y - y_d}_{\lspace}^2 + \alpha \norm{q}_{L^1(\Omega,L^2(I))} + \frac{1}{\gamma}\norm{q}_{L^2(\Omega,L^2(I))}.
\label{costreg}
\ee
The parameter $\frac{1}{\gamma}$ shall be chosen sucht that it is small in comparison to $\alpha$. This enables us to look for controls in the space $L^2(I\times\Omega)$. Since the embedding $L^1(\Omega,L^2(I))\hookrightarrow \M$ is isometric, then any $y \in L^1(\Omega,L^2(I))$ will satisfy
\be
\norm{y}_{\M} = \norm{y}_{L^1(I,L^2(\Omega))} = \int_0^L{\norm{y}_{L^2(I)}dx}.
\ee
In \cite{herzog2012directional}, the authors study this problem in the linear case, and prove that this setting promotes a \textit{striped sparsity pattern}. In the nonlinear case, the development is analogous and we can adapt it to state the following theorem
\begin{thm}
 Let $\gamma > 0$. Problem \eqref{costreg} possesses a unique optimal solution  $q_{\gamma} \in L^2(I\times\Omega)$ with corresponding state $y_{\gamma} = S(y_0,q_{\gamma})$ and adjoint state $p_{\gamma} = S'^{\ast}(y_{\epsilon} - y_d)$. The first order optimality condition is the subgradient condition
 \be
 -<q - q_{\gamma}, \frac{1}{\gamma}q_{\gamma} + p_{\gamma}> +\alpha \norm{q_{\gamma}}_{L^1(I,L^2(\Omega))} \leq \alpha \norm{q}_{L^1(I,L^2(\Omega))}
 \ee
 for all $q \in L^1(I,L^2(\Omega))$. This is summed up in the formula
 \be
 q_{\gamma}(t,x) = \gamma\max\left(0,1-\frac{\alpha}{\norm{p_{\gamma}}_{L^2(I)}}\right)p_{\gamma}(t,x)
 \ee
 for almost all $(t,x)\in I\times\Omega$.
\end{thm}
The idea is then to solve with a semismooth Newton method \cite{ulbrich2002semismooth} the equation
\be
F(q_{\gamma}) = q_{\gamma} - \gamma\max\left(0,1-\frac{\alpha}{\norm{p_{\gamma}}_{L^2(I)}}\right)p_{\gamma}(t,x)
\ee
In comparison to what is done in \cite{herzog2012directional}, the nonlinearity in our problem requires that we additionally compute the tangent and the second adjoint operators to get the gradient of $F$. Besides, in \cite{pieper2014}, the authors quickly prove that we obtain the original problem in the limiting case for great $\gamma$. 


\subsection{Numerical examples}
In this section we consider two kinds of problem that can be encountered in practice in the context of water wave:


%%% Local Variables: 
%%% mode: latex
%%% TeX-master: "kdv.tex"
%%% End: