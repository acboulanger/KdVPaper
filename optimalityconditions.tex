%!TEX root = kdv.tex
%%%%%%%%%%%%%%%%%%%%%%%%%%%%%%%%%%%%%%%%%%%%%%%%%
\section{First order optimality conditions}
%%%%%%%%%%%%%%%%%%%%%%%%%%%%%%%%%%%%%%%%%%%%%%%%%
\label{secoptconditions}
Next we discuss the differentiability of the control-to-state operator $S$ defined in \eqref{controltostate}.
\begin{prop}
  The control to state operator $S$ is continuously Fr\'echet-differentiable. Its derivative
  \[
  S'(u)\colon \M\rightarrow \mathcal B,~\delta u\mapsto
  \delta y
  \]
  at $u\in U_{ad}$ is given by the solution operator of the linear tangent equation
  \begin{subequations}
   \begin{numcases}{}
      \partial_t \delta y +\partial_x \delta y -\gamma \partial_{xx} \delta y+ \partial_{xxx} \delta y + \partial_x (y \delta y)=\delta u \mbox{ in } I\times\Omega,  \label{linearizedkdv1}\\
      \delta y(\cdot,0) = \delta y(\cdot,L) = \partial_x \delta y (\cdot,L) = 0 \mbox{ on } I,  \label{linearizedkdv2}\\
      \delta y(0,\cdot) = 0 \mbox{ on } \Omega  \label{linearizedkdv3}.
   \end{numcases}
   \label{tan eq}
  \end{subequations}
  \label{propfrechet}
\end{prop}
\begin{proof}[Proof of Proposition~\ref{propfrechet}]
First we mention that the non-linearity $F:\mathcal B\rightarrow L^2(I,H^{-1}(\Omega)),\quad y\mapsto y\partial_x y$ is Fr\'echet differentiable since with $F'(y)\delta y=\partial_x(y\delta y)$ for any $\delta y\in \mathcal B\hookrightarrow L^4(I\times \Omega)$
\[
\|F(y+\delta y)-F(y)-F'(y)\delta y\|_{\Hm1}\leq \frac 1 2\|\delta y\|_{L^4(I\times \Omega)}^2\leq c\,\|\delta y\|_{\mathcal B}^2.
\]
Then we differentiate the fixed point equation $y=\mathcal L(u-y\partial_x y,y_0)$ with respect to $(y,u)$ in direction $(\delta y,\delta u)\in \mathcal B\times \M$ and get
\be\label{abstract_tangent_equation}
\delta y=\mathcal L'(\delta u-\partial_x(y\delta y))
\ee
where $\mathcal L'(\cdot)=\mathcal L(\cdot,0)$. In the Appendix~\ref{appendixtangent} it is shown that \eqref{abstract_tangent_equation} has a unique solution $\delta y\in \mathcal B$.  Actually $\delta y$ is the weak solution of \eqref{linearizedkdv1}-\eqref{linearizedkdv3} in the sense of \eqref{weakformkdv}. Next we show that $S'(u)\delta u:=\delta y$ is the Fr\'echet derivative of $S$. This will result from the study of
\[
\frac{1}{\norm{\delta u}_{\M}} \norm{S(u + \delta u) - S(u)  -S'(u)\delta u}_{\mathcal B} = \frac{1}{\norm{\delta u}_{\M}} \norm{\tilde{y} - y  -\delta y}_{\mathcal B}.
\]
Calling $w = \tilde{y} - y - \delta y\in \mathcal B$ the function $w$ then satisfies
\bean
  &\partial_t w +\partial_x w + \partial_{xxx} w - \gamma \partial_{xx} w  + \tilde{y}\partial_x \tilde{y}  - y\partial_x y - \partial_x(y\delta y)=  0 \mbox{ in } I\times\Omega,\nonumber\\
  &w(.,0) = w(.,L) = \partial_x w (.,L) = 0 \mbox{ in } I,\nonumber\\
  &w(0,x) = 0 \mbox{ in } \Omega\nonumber.
\eean
in the weak sense of \eqref{weakformkdv}. After rearranging the terms we end up with
\besn
\partial_t w +\partial_x w + \partial_{xxx} w - \gamma \partial_{xx} w  + \partial_x(yw) =  -(\tilde{y} - y)\partial_x(\tilde{y} - y) \mbox{ in } I\times\Omega,\label{kdvw1}\\
w(.,0) = w(.,L) = \partial_x w (.,L) = 0 \mbox{ in } I,\label{kdvw2}\\
w(0,x) = 0\mbox{ in } \Omega\label{kdvw3}.
\eesn
According to Appendix~\ref{appendixtangent} and Lemma \ref{lemyyx2} it holds
\[
\norm{w}_{\mathcal{B}} \leq C\norm{(\tilde{y} - y)\partial_x(\tilde{y} - y)}_{\Hm1}\leq C T^{1/4}\norm{\tilde{y} - y}_{\mathcal{B}}.
\]
Therefore the conclusion follows from Lipschitz continuity of $S$. Hence we provide the following lemma which concludes the proof.
\end{proof}
\begin{lem}
Let $u\in U_{ad}$ and $u+\delta u\in U_{ad}$ for $\delta u\in \M$. The control-to-state operator $S$ satisfies the estimate
\[\|S(u+\delta u)-S(u)\|_{\mathcal B}\leq C(\|u\|_{\M})\|\delta u\|_{\M}.\]
 \label{lipschitzcontinuity}
\end{lem}
\begin{proof}[Proof of Lemma~\ref{lipschitzcontinuity}]
We define $y = S(u) \in \mathcal{B}$ and $\tilde{y} = S(u+\delta u) \in \mathcal{B}$. Therefore, the difference $w = \tilde{y} - y$ satisfies the equation
\begin{subequations}
 \begin{numcases}{}
    \partial_t w +\partial_x w + \partial_{xxx} w - \gamma \partial_{xx} w  + \partial_x (\bar{y}w)=  \delta u \mbox{ in } I\times\Omega,\label{diffkdv4}\\
    w(.,0) = w(.,L) = \partial_x w (.,L) = 0 \mbox{ in } I,\label{diffkdv5}\\
    w(0,x) = 0 \mbox{ in } \Omega\label{diffkdv6},
 \end{numcases}
\end{subequations}
with $\bar{y} = \frac{\tilde{y} + y}{2} \in \mathcal{B}$ in the weak sense of \eqref{weakformlinearkdv}. According to Appendix \ref{appendixtangent} $w \in \mathcal{B}$ satisfies the a priori estimates
\[
\norm{w}_{\mathcal{B}} \leq C(T,L,\|u\|_{\M})\norm{\delta u}_{\M}.
\]
% We have
% \beal
%   \norm{\bar y}_{\mathcal{B}} &\leq \frac{1}{2} \left( \norm{y}_{\mathcal{B}} + \norm{\tilde{y}}_{\mathcal{B}}\right)\\
%   & \leq \frac{1}{2} C(T,L)\left( \norm{q+\delta q}_{L^2(I,H^{-1}(\Omega))} + \norm{q}_{L^2(I,H^{-1}(\Omega))} \right)\\
%   & \leq C(T,L)\norm{q}_{L^2(I,H^{-1}(\Omega))} +\frac{C(T,L)}{2}\norm{\delta q}_{L^2(I,H^{-1}(\Omega))}
% \eeal
Thus we can conclude that the solution operator $S$ is locally Lipschitz continuous.
\end{proof}
%   First we define the nonlinear operator \be E\colon S\times \mathcal
%   U_{ad}\rightarrow L^2(I,H^{-2}(\Omega))\times L^2(\Omega) \ee wit
%   $S:=\{y\in L^2(I,H^1_0(\Omega))\cap \mathcal C(\bar
%   I,L^2(\Omega))\colon y_x(\cdot,L)=0\}\cap H^1(I,H^{-2}(\Omega))$
%   which consists of \be E(y,q) = L(y,q)+H(y) \ee with \be
%   L(y,q)=\left(
%     \begin{array}{c}
%       \partial_t y+\partial_x y-\gamma \partial_{xx}y+\partial_{xxx} y-q\\
%       y(0,\cdot)
%     \end{array}
%   \right),~~ H(y)=\left(
%     \begin{array}{c}
%       y\partial_x y\\
%       0
%     \end{array}
%   \right).  \ee The operator $E$ allows us to write the state-equation
%   in the following form \be E(y,q)=0,~y\in S,~q\in U_{ad}.  \ee The
%   linear terms in $L$ are continuously Fr\'echet-differentiable. The
%   nonlinear term \be H_1\colon S\rightarrow
%   L^2(I,H^{-1}(\Omega)),~y\mapsto y\partial_x y \ee is also
%   continuously Fr\'echet-differentiable with the derivative
%   $H'_1(y)\delta y=\partial_x(y\delta y)$. This is true since it holds
%   \be \|H_1(y+\delta y)-H_1(y)-H'_1(y)\delta
%   y\|_{L^2(I,H^{-1}(\Omega))}=\|\delta y\partial_x \delta
%   y\|_{L^2(I,H^{-1}(\Omega))}\leq c\|\delta
%   y\|_{L^2(I,H^1_0(\Omega))}\|\delta y\|_{\mathcal C(\bar
%     I,L^2(\Omega))}, \ee for $y,\delta y\in S$ according to
%   Lemma~\ref{lemyyx2}. This means that $E$ is continuously
%   Fr\'echet-differentiable. The partial derivative of $E$ with respect
%   to $y$ is given by \be E'_y(y,q)\delta y=\left(
%     \begin{array}{c}
%       \partial_t \delta y+\partial_x \delta y-\gamma \partial_{xx} \delta y+\partial_{xxx}\delta y + \partial_x(y\delta y)\\
%       \delta y(0,\cdot)
%     \end{array}
%   \right) \ee and with respect to $q$ by $E'_q(y,q)\delta q=-\delta
%   q\in \M$. Therefore it is enough that the operator $E'_y(y,q)\colon
%   S\rightarrow L^2(I,H^{-2}(\Omega))\times L^2(\Omega)$ is
%   continuously invertible from $S$ to $L^2(I,H^{-1}(\Omega))$. Indeed,
%   the equation \be E'_y(y,q)\delta y=(f,y_0)^t \ee has a unique
%   solution $\delta y\in S$ for any $(f,y_0)\in
%   L^2(I,H^{-1}(\Omega))\times L^2(\Omega)$ and fulfills the following
%   a priori estimate \be \|\delta y\|_{L^2(I,H^1_0(\Omega))}+\|\delta
%   y\|_{\mathcal C(\bar I,L^2(\Omega))} \leq
%   c(\|y\|_{L^2(I,H^1_0(\Omega))})\left(\|f\|_{L^2(I,H^{-2}(\Omega))}+\|y_0\|_{L^2(\Omega)}\right)
%   \ee according to Theorem ??. So the implicit function theorem
%   \cite{pinnau2008optimization} guarantees the existence of a
%   continuously Fr\'echet differentiable mapping \be M\colon
%   U_{ad}\rightarrow S,~~q\mapsto y \ee which fulfills \be E(M(q),q)=0.
%   \ee Its derivative \be M'(q)\colon \M\rightarrow S,~~\delta q\mapsto
%   \delta y,~q\in U_{ad} \ee is given by \be M'(q)\delta
%   q=-E'_y(M(q),q)^{-1}\delta q,~~\delta q\in \M,~q\in U_{ad}.  \ee

% \begin{rmk}
%   \bealn
%   &\partial_t \delta y +\partial_x \delta y - \gamma \partial_{x} \delta y+\partial_{xxx} \delta y + \partial_x(y\delta y)=\delta q \mbox{ in } \Omega,\\
%   &\delta y(.,0) = \delta y(.,L) = \partial_x \delta y (.,L) = 0 \mbox{ in } \Gamma,\\
%   &\delta y(0,.) = 0 \mbox{ on } \Omega.  \eealn is called linearized
%   state equation or tangent equation.
% \end{rmk}
Therefore the control-to-observation operator $S_{obs}$ is also Fr\'echet differentiable and its derivative is given by
\[S_{obs}'(u)\colon \delta u\mapsto (\chi_{\Omega_o}\delta y,\chi_{\Omega_o}\delta y(T))\]
where $\delta y\in \mathcal B$ is the weak solution of \eqref{tan eq}. Next we introduce we adjoint control to observation operator $S'^\star$.
\begin{prop}
  Let $u\in U_{ad}$. There exists a linear operator
  \[
  S'^\star(u)\colon L^2(I\times\Omega_o)\times L^2(\Omega_o)\rightarrow \C, (\phi,p_T)\mapsto p
  \]
  which fulfills
  \begin{multline}
  (S'(u)\delta u,\phi)_{L^2(I\times \Omega_o)}+((S'(u)\delta u)(T),p_T)_{L^2(\Omega_o)}=\langle\delta u, S'^\star(u)(\phi,p_T)\rangle_{\M,\C}\\
  \forall \delta u\in\M,~(\phi,p_T) \in L^2(I\times \Omega_o)\times L^2(\Omega_o).
  \label{adjointwf}
  \end{multline}
  Moreover it is the solution operator of
  \begin{subequations}
   \begin{numcases}{}
      -\partial_t p -\partial_x  p -\gamma \partial_{xx} p - \partial_{xxx} p  - y\partial_x p=\phi \mbox{ in } I\times\Omega,\label{adjointKdV1}\\
      p(\cdot,0) = p(\cdot,L) = \partial_x p(\cdot,0) = 0 \mbox{ in } I,\label{adjointKdV2}\\
      p(T,\cdot) = p_T \mbox{ in } \Omega\label{adjointKdV3}
   \end{numcases}
  \end{subequations}
  with $y=S(u)$.
  \label{adjointKdV}
\end{prop}
\begin{proof}[Proof of Proposition~\ref{adjointKdV}]
First of all we mention that $y\partial_x p\in L^1(I,L^2(\Omega))$ holds for $y\in \mathcal B$ and $p\in \mathcal B$, c.f., Lemma \ref{lemadjoint}. Then we consider the weak formulation of the tangent equation
\be
(\delta y,\phi)_{L^2(I\times\Omega)}+(\delta y(T),p_T)_{L^2(\Omega)}-(\delta y,y\partial_x p)_{L^2(I\times\Omega)}=\langle \delta u,p\rangle_{\M,\mathcal C_I},\quad\delta u\in \M
\label{tangentwf}
\ee
with $y=S(u)\in\mathcal B$ and therefore  $\delta y=S'(u)\delta u\in\mathcal B$. By identification with \eqref{adjointwf}, we set $S'^\star(u)(\phi,p_T):=p$ where $p$ solves the fixed point equation
\be\label{fixed_point_equation_adjoint}
p(t)=W^*(T-t)p_T+\int_t^TW^*(s-t)(\phi(s)-y(s)\partial_xp(s))~\mathrm ds,~t\in I,
\ee
derived from the form \eqref{adjointmild} with a source term that now also depends on $p$.
The fixed point equation \eqref{fixed_point_equation_adjoint} has a unique solution $p\in \mathcal B\hookrightarrow\C$, for a proof see Appendix \ref{appendixadjoint}.
\end{proof}
Next we derive first order optimality conditions using tools from convex analysis.
\begin{prop}
  Let $(\bar y,\bar u)\in \mathcal B\times U_{ad}$ be the solution of \eqref{cost}. Then there
  exists a unique $\bar p\in \C$ which is the solution of the adjoint state equation
  \bean
  &-\partial_t\bar p -\partial_x \bar p -\gamma \partial_{xx} \bar p -\partial_{xxx} \bar p - \bar y\partial_x \bar p=\chi_{\Omega_o}\bar y-z_1 \mbox{ in } \Omega,\\
  &\bar p(\cdot,0) = \bar p(\cdot,L) = \partial_x \bar p(\cdot,0) = 0 \mbox{ in } I,\\
  &\bar p(T,\cdot) =\chi_{\Omega_o}y(T)-z_2 \mbox{ in } \Omega.
  \eean
  and fulfills the following variational inequality condition
  \be
  \langle -\bar p,u-\bar u\rangle_{\C,\M}+\|\bar u\|_{\M}\leq\|u\|_{\M}\quad\forall u\in U_{ad}.
  \label{subgradientcond}
  \ee
\end{prop}
\begin{proof}
  We define $G(u)=(1/2)\|S_{obs}(u)-z\|_{L^2(I\times \Omega_{obs})}^2$ and $F(u)=\|u\|_{\M}$+$\mathcal I_{U_{ad}}(u)$. Then the directional derivative of $G$ can be expressed using the operator $S_{obs}'^\star(\bar u)$ namely
  \[
  \nabla G(\bar u)=S_{obs}'^\star(\bar u)(S_{obs}(\bar u)-z)\in \C.
  \]
  We define $\bar p := \nabla G(\bar u)$. An element $\bar u\in U_{ad}$ is optimal if
  \[
  G(\bar u)+F(\bar u)\leq G(u)+F(u)\quad\forall u\in U_{ad}
  \]
  and in  particular
  \[
  G(\bar u)+F(\bar u)\leq G(\bar u + \varepsilon(u-\bar u))+F(\bar u+ \varepsilon(u-\bar u))
  \]
  for some $0<\varepsilon$ small enough such that $\bar u + \varepsilon(u-\bar u)\in U_{ad}$ holds. Using the convexity of $F$ we get
  \[
    \frac{G(\bar u)-G(\bar u + \varepsilon(\delta u-\bar u))}{\varepsilon}+ F(\bar u)\leq F(u)
  \]
  which implies
  \[
  \langle-\bar p ,u-\bar u\rangle_{\C,\M} + F(\bar u)\leq F(u)\quad\forall u\in U_{ad}.
  \]
\end{proof}
\begin{rmk}
The subgradient condition \eqref{subgradientcond} is equivalent to 
\begin{equation}\label{equivoc}
\alpha \|\bar u\|_{\M}+\phi^{\star}(-\bar p)=\langle -\bar p,\bar u\rangle_{\C,\M},\quad \bar u\in U_{ad}
\end{equation}
with
\[
\phi^{\star}(p)=\sup_{u\in U_{ad}}[\langle u,p\rangle_{\C,\M}-\alpha \|u\|_{\M}]
\]
for $p\in \C$.
\end{rmk}
\begin{lem}
The functional $\phi^{\star}(p)\colon \C\rightarrow \mathbb R$ has the form
\[
\phi^{\star}(p)=\hat c\,(\|p\|_{\C}-\alpha))^+
\]
where $\hat c = C/T^{1/4}-\|y_0\|_{L^2(\Omega)}$ and $(\cdot)^+ = \operatorname{max}(0,\cdot)$ is the positive part of a function.
\end{lem}
\begin{proof}
\begin{align*}
\phi^{\star}(p)&=\sup_{u\in U_{ad}}[\langle u,p\rangle_{\C,\M}-\alpha \|u\|_{\M}]\\
&=\sup_{\|u\|_{\M}=1,\,\lambda\in [0,c]}\lambda[\langle u,p\rangle_{\C,\M}-\alpha].
\end{align*}
This implies the assertion since $\|p\|_{\C}=\sup_{\|u\|_{\M}=1}\langle u,p\rangle_{\C,\M}$.
\end{proof}
\begin{rmk}
A vector measure $u\in \M$ can be decomposed in a space dependent measure part $|u|\in \mathcal M(\Omega)^+$ and a space-time dependent function $u'\in  L^1((\Omega,|u|),L^2(I))$ which satisfies $\|u(x)\|_{L^2(I)}=1$ for $|u|$-a.e. $x\in \Omega$. In particular it holds
  \[
  \mathrm du=u'~\mathrm d|u|.
  \]
The duality pairing between $\C$ and $\M$ takes the form
\[
\langle u,p\rangle_{\M,\C}=\int_\Omega(u',p)_{L^2(I)}~\mathrm d|u|.
\]  
\end{rmk}
In order to handle the the norm constraint we need the following assumption. Based on this assumption we can show the following result.
\begin{prop}
Let $\bar u\in U_{ad}$ be the optimal control of \eqref{cost}. Moreover let $|\bar u|\in \mathcal M(\Omega)$ be its total-variation measure and $\bar u'$ its the Radon-Nikodym-derivative. Furthermore $\bar p\in \C$ be the corresponding optimal adjoint state. Then there holds
\begin{align}
0&=\left(\|\bar u\|_{\M}-\hat c\right)\bar\lambda\label{scalarlagrange}\\
\operatorname{supp}|\bar u|&\subseteq \{x\in \Omega\colon \|\bar p\|_{L^2(I)}=\alpha+\bar\lambda\}\label{controlsupp}.
\end{align}
with $\bar\lambda =(\|\bar p\|_{\C}-\alpha)^+$. Moreover we have
\be
\bar u'(x)=-\frac{\bar p(x)}{\alpha +\bar\lambda}\quad\text{in}~L^1((\Omega,|\bar u|),L^2(I)).
\label{controlintime}
\ee
\label{propsubgcondition}
\end{prop}
\begin{proof}
Using $\bar u\in U_{ad}$, $\|\bar p\|_{\C}-\bar\lambda\leq\alpha$ and \eqref{equivoc} we can estimate
\begin{multline}\label{estoc}
\alpha\|\bar u\|_{\M}=\int_{\Omega}(\bar u',\bar p)_{L^2(I)}~\mathrm d|\bar u|-\hat c\bar\lambda\leq \int_\Omega(\bar u',\bar p)_{L^2(I)}~\mathrm d|\bar u|-\bar\lambda\|\bar u\|_{\M}\\
\leq \int_\Omega\|\bar u'\|_{L^2(I)}\|\bar p\|_{L^2(I)}~\mathrm d|\bar u|-\bar\lambda\|\bar u\|_{\M}\\
\leq\|\bar u\|_{\M}(\|\bar p\|_{\C}-\bar\lambda)\leq \alpha\|\bar u\|_{\M}.
\end{multline}
Thus the last chain of inequalities holds with equality and therefore we have
\[
\int_{\Omega}(\bar u',\bar p)_{L^2(I)}-\|\bar u'\|_{L^2(I)}\|\bar p\|_{L^2(I)}~\mathrm d|\bar u|=0
\]
which implies $\bar u'=-(1/\|\bar p\|_{L^2(I)})\,\bar p$ in $L^1((\Omega,|\bar u|),L^2(I))$. Moreover we have
\[
\int_\Omega\|\bar p\|_{L^2(I)}-\bar\lambda-\alpha~\mathrm d|\bar u|=0
\] 
which then implies \eqref{controlsupp} and \eqref{controlintime}. Equality in \eqref{estoc} also implies
\[
(\hat c-\alpha\|\bar u\|_{\M})\bar\lambda=0.
\]   
\end{proof}
\begin{rmk}
If the constraint $\|\bar u\|_{\M}\leq \hat c$ is not active there holds $\bar \lambda=0$ and we recover the usual optimality conditions; c.f. \cite{pieper2014}. This can be guaranteed for a large enough parameter $\alpha$.
\end{rmk}
%\begin{proof}[Proof of Proposition~\ref{propsubgcondition}]
%
%  %The subgradient condition \eqref{subgradientcond} is equivalent to
%%  \be
%%  \|\bar q\|_{\M}+(\|\cdot\|_{\M})^*(-\bar p)=\langle \bar q,-\bar p\rangle_{\M,\M^*}
%%  \label{subgradientcondfenchel4}
%%  \ee
%%  where
%%  \[
%%  (\|\cdot\|_{\M})^*(-\bar p)=\begin{cases} 0~~&\|\bar p\|_{\M^*}\leq\alpha\\ \infty~~&\text{else}\end{cases}
%%  \]
%%  is the convex conjugate functional of $\|\cdot\|_{\M}$. Then
%%  \eqref{subgradientcondfenchel4} is equivalent to
%%  \bean
%%  &\|\bar q\|_{\M}+\langle\bar q,\bar p\rangle_{\M,\M^*}=0 \\
%%  &\|\bar p\|_{\M^*}\leq\alpha.
%%  \eean
%%  Recall that $\bar p=\nabla G(\bar q)\in \C\subset \M^*$. This and $C_0(\Omega,L^2(I))^* = \M$ imply
%%  \[
%%  \|\bar p\|_{\M^*}=\|\bar p\|_{\C}
%%  \]
%%  which yields the assertion.
%\end{proof}


%\begin{proof}[Proof of Theorem~\ref{supportcontrol}]
%  Let $\bar q'$ be the density of $\bar q$ with resect to $|\bar
%  q|$. From \eqref{subgradientconditionfenchel1} - \eqref{subgradientconditionfenchel2} follows
%  \begin{equation}\label{supp_opt_control_estimate}
%    \begin{aligned}
%      \alpha \|\bar q\|_{\M} &= \langle \bar q,-\bar p \rangle_{\M,C_0(\Omega,L^2(I))}\\
%      & = \int_{\Omega}{(\bar q'(x),-\bar p(x))_{L^2(I)}~\mathrm{d}|\bar q|} \\
%      & \leq \int_{\Omega}{\|\bar q'(x)\|_{L^2(I)}\|\bar p(x)\|_{L^2(I)}~\mathrm d|\bar q}|\\
%      & \leq \|\bar q\|_{\M} \|\bar p\|_{C_0(\Omega,L^2(I))}\\
%      & \leq \alpha \|\bar q\|_{\M}.
%    \end{aligned}
%  \end{equation}
%  This means that \eqref{supp_opt_control_estimate} holds with
%  equality, in particular it holds
%  \begin{equation}\label{complementarity_optimal_control}
%    \int_\Omega \underbrace{\|\bar p(x)\|_{L^2(I)}-\alpha}_{\leq 0}~\mathrm d|q|=0
%  \end{equation}
%  which implies
%  \begin{equation}\label{norm_alpha}
%    \|\bar p(x)\|_{L^2(I)}=\alpha~~|\bar q|-\text{a.e.}~x\in \Omega
%  \end{equation}
%  and \eqref{support_optimal_control} due to the continuity of
%  $\alpha-\|\bar p(\cdot)\|_{L^2(I)}$.  From equality in
%  \eqref{supp_opt_control_estimate} also follows
%  \begin{equation*}
%    \int_\Omega\underbrace{\|\bar q'(x)\|_{L^2(I)}\|\bar p(x)\|_{L^2(I)}-(\bar q'(x),-\bar p(x))_{L^2(I)}}_{\geq0}~\mathrm{d}|\bar q|=0.
%  \end{equation*}
%  Due to the sign of the integrant it even holds
%  \begin{equation*}
%    \|\bar q'(x)\|_{L^2(I)}\|\bar p(x)\|_{L^2(I)}-(\bar q'(x),-\bar p(x))_{L^2(I)}=0~~|\bar u|-\text{a.e.}~x\in \Omega
%  \end{equation*}
%  which means that $q'$ and $-\bar p$ are colinear. In particular,
%  $\|q'(\cdot)\|_{L^2(I)}\equiv1$ and \eqref{norm_alpha} imply
%  \begin{equation}\label{supp_opt_control_representation}
%    \bar q'(x,t)=-\frac{1}{\alpha}\bar p(x,t)~~|\bar q|-\text{a.e.}~x\in\Omega,~\text{a.e.}~t\in I.
%  \end{equation}

%%% Local Variables:
%%% mode: latex
%%% TeX-master: "kdv.tex"
%%% End:
